%\title{LaTeX Portrait Poster Template}
%%%%%%%%%%%%%%%%%%%%%%%%%%%%%%%%%%%%%%%%%
% a0poster Portrait Poster
% LaTeX Template
% Version 1.0 (22/06/13)
%
% The a0poster class was created by:
% Gerlinde Kettl and Matthias Weiser (tex@kettl.de)
% 
% Adapter by Jens Buysse for Hogeschool Gent
% This template has been downloaded from:
% http://www.LaTeXTemplates.com
%
% License:
% CC BY-NC-SA 3.0 (http://creativecommons.org/licenses/by-nc-sa/3.0/)
%
%%%%%%%%%%%%%%%%%%%%%%%%%%%%%%%%%%%%%%%%%

%----------------------------------------------------------------------------------------
%	PACKAGES AND OTHER DOCUMENT CONFIGURATIONS
%----------------------------------------------------------------------------------------
\documentclass[a0,portrait]{a0poster}
\usepackage{multicol}
\columnsep=100pt
\columnseprule=3pt
\usepackage[svgnames]{xcolor}
\usepackage{times}
\usepackage{graphicx}
\graphicspath{{figures/}}
\usepackage{booktabs}
\usepackage[font=small,labelfont=bf]{caption}
\usepackage{amsfonts, amsmath, amsthm, amssymb}
\usepackage{wrapfig}
\usepackage[export]{adjustbox}

\begin{document}

%----------------------------------------------------------------------------------------
%	POSTER HEADER 
%----------------------------------------------------------------------------------------

\begin{minipage}[t]{0.75\linewidth}
\VeryHuge \color{HoGentAccent1} \textbf{RPA, automatisatie van morgen, vandaag} \color{Black}\\
\huge \textbf{Pessemier Mout, Lavaert Laurens, Decorte Johan}\\[0.5cm]
\huge Hogeschool Gent, Valentin Vaerwyckweg 1, 9000 Gent\\[0.4cm]
\Large \texttt{mout.pessemier@hogent.be} \\
\end{minipage}
%
\begin{minipage}[t]{0.25\linewidth}
\includegraphics[width=13cm,right]{figures/HOGENT_Logo_Pos_rgb.png} 

\end{minipage}

\vspace{1cm}

%----------------------------------------------------------------------------------------

\begin{multicols}{2}

%----------------------------------------------------------------------------------------
%	ABSTRACT
%----------------------------------------------------------------------------------------

\color{HoGentAccent1}

\begin{abstract}
	Robotic Process Automation is een innovatieve sector, dé oplossing voor efficiënt automatiseren van processen binnen een bedrijf. Door het elimineren van repetitieve taken, kan de focus gelegd worden op essentiëlere werkopdrachten. Dit impliceert een beduidend lagere foutenlast en de werkkwantiteit van het bedrijf vermeerderd aanzienlijk. 
\end{abstract}

%----------------------------------------------------------------------------------------
%	INTRODUCTION
%----------------------------------------------------------------------------------------
\color{HoGentAccent1} 
\section*{Introductie}
\color{black}
In de bedrijfswereld wordt een sterke nadruk gelegd op het efficiënt uitvoeren van processen. Een manier om de efficiëntie te verhogen is het automatiseren van deze bedrijfsprocessen. Robotic Process Automation (RPA) is een manier om deze automatisering te verwezenlijken. Nu reist enkel nog de vraag, welk platform? In de zee van RPA-providers is het niet gemakkelijk om de juiste provider te kiezen die past bij uw bedrijf. Er moet onder andere rekening gehouden worden met de prijs, de community, AI capaciteiten en de mogelijkheden om eigen integraties toe te voegen.\\
Tijdens dit onderzoek zijn onder andere deze eigenschappen onderzocht geweest zodat de keuze voor een bepaald platform makkelijker verloopt. Deze keuze wordt gefaciliteerd door onder andere het uitwerken van een proces op de verschillende platformen, de communicatie met de diverse bedrijven achter de platformen en een marktonderzoek.

%----------------------------------------------------------------------------------------
%	GEOLOGY
%----------------------------------------------------------------------------------------
\color{Black}
\color{HoGentAccent1} 
\section*{Experimenten}
\color{black}
Het uitgevoerde experiment bestaat uit een voorbeeldproces tot automatiseren op de verschillende onderzochte RPA-platformen. De vooropgestelde activiteiten waaraan de provider minstens moet voeldoen, is de uitvalbasis voor dit proces. De nadruk werd gelegd op het werken met het bestandssysteem van de host, het versturen van een mail en het voorzien van een zelfgeschreven integratie.\\
Voor elk platform werd het proces uitgewerkt en de eigen activiteit geïntegreerd. Nadien werd er extensief getest. Er werd ook onderzoek gedaan naar het verschil tussen de gratis versie en de premium versie, het forum, de community en de prijs. Naderhand werden aan elke provider quotaties toegekend volgens de bevindingen.

%----------------------------------------------------------------------------------------
%	CONCLUSION
%----------------------------------------------------------------------------------------
\color{HoGentAccent1} 
\section*{Conclusies}
\color{black}
Ruw gezien omvatten de verschillende platformen dezelfde functionaliteiten. Dit bevestigd nog maar eens de moeilijkheid om de juiste provider te kiezen. Een algemene conclusie die kan getrokken worden is de eerder prijzige kant van RPA. Er is ook een duidelijke race aan de gang tussen 'The Big Three'. Voor de onderzochte providers, komt UiPath als duidelijke winnaar uit de bus. Dit platform heeft de grootste community en wordt het meest gebruikt. Het marktonderzoek staaft dan ook deze bevindingen. De eervolle tweede plaats gaat naar Automation Anywhere. Hun online platform om bots te onderhouden heeft minder mogelijkheden. Daarenboven kwam IntelliBot, als keine provider, lovenswaardig uit het onderzoek. Eerlijkheidshalve is er ook een keerzijde van de medaille. WorkFusion heeft nog een lange weg te gaan alvorens ze ooit kunnen concurreren. Ze scoren zwak op hun platform, de community en de tools die ze voorzien. Ook bij Microsoft Flow reist enige vorm van kritiek. De werkwijze en mogelijkheden van het platform scoren ondermaats.

%----------------------------------------------------------------------------------------
%	FORTHCOMING RESEARCH
%----------------------------------------------------------------------------------------
\color{HoGentAccent1} 
\section*{Toekomstig onderzoek}
\color{black}
Uit het ruime aanbod van providers zijn er slechts enkele uitgekozen die onderzocht werden. Mogelijks brengt verder onderzoek naar de andere RPA-providers nieuwe kandidaten naar boven. Ze zullen wel uit hun pijp moeten komen om te kunnen concurreren met 'The Big Three'.Dieper onderzoek in de RPA markt kan interessante informatie leveren over de toekomst en de richting dat deze technologie kan inslaan.

%----------------------------------------------------------------------------------------

\end{multicols}
\end{document}