%\title{LaTeX Portrait Poster Template}
%%%%%%%%%%%%%%%%%%%%%%%%%%%%%%%%%%%%%%%%%
% a0poster Portrait Poster
% LaTeX Template
% Version 1.0 (22/06/13)
%
% The a0poster class was created by:
% Gerlinde Kettl and Matthias Weiser (tex@kettl.de)
% 
% Adapter by Jens Buysse for Hogeschool Gent
% This template has been downloaded from:
% http://www.LaTeXTemplates.com
%
% License:
% CC BY-NC-SA 3.0 (http://creativecommons.org/licenses/by-nc-sa/3.0/)
%
%%%%%%%%%%%%%%%%%%%%%%%%%%%%%%%%%%%%%%%%%

%----------------------------------------------------------------------------------------
%	PACKAGES AND OTHER DOCUMENT CONFIGURATIONS
%----------------------------------------------------------------------------------------
\documentclass[a0,portrait]{a0poster}
\usepackage{multicol}
\columnsep=100pt
\columnseprule=3pt
\usepackage[svgnames]{xcolor}
\usepackage{times}
\usepackage{graphicx}
\graphicspath{{figures/}}
\usepackage{booktabs}
\usepackage[font=small,labelfont=bf]{caption}
\usepackage{amsfonts, amsmath, amsthm, amssymb}
\usepackage{wrapfig}
\usepackage[export]{adjustbox}

\begin{document}

%----------------------------------------------------------------------------------------
%	POSTER HEADER 
%----------------------------------------------------------------------------------------

\begin{minipage}[t]{0.75\linewidth}
\VeryHuge \color{HoGentAccent1} \textbf{RPA, automatisatie van morgen, vandaag} \color{Black}\\
\huge \textbf{Pessemier Mout, Lavaert Laurens, Decorte Johan}\\[0.5cm]
\huge Hogeschool Gent, Valentin Vaerwyckweg 1, 9000 Gent\\[0.4cm]
\Large \texttt{mout.pessemier@hogent.be} \\
\end{minipage}
%
\begin{minipage}[t]{0.25\linewidth}
\includegraphics[width=13cm,right]{figures/HOGENT_Logo_Pos_rgb.png} 

\end{minipage}

\vspace{1cm}

%----------------------------------------------------------------------------------------

\begin{multicols}{2}

%----------------------------------------------------------------------------------------
%	ABSTRACT
%----------------------------------------------------------------------------------------

\color{HoGentAccent1}

\begin{abstract}
	Robotic Process Automation is een innovatieve methode om bedrijfsprocessen efficiënt te automatiseren. Door het elimineren van repetitieve taken, kunnen werknemers zich meer focussen op complexere uitdagingen. Bovendien worden deze taken sneller en met minder fouten uitgevoerd, wat de personeelskost enorm reduceert.
\end{abstract}

%----------------------------------------------------------------------------------------
%	INTRODUCTION
%----------------------------------------------------------------------------------------
\color{HoGentAccent1} 
\section*{Introductie}
\color{black}
Efficiëntie is een sleutelconcept voor ondernemingen. Menselijke fouten zijn vrijwel onvermijdelijk en ook de performantie van werknemers is begrensd. Controlesystemen en correcties zorgen daarbij voor bijkomende kosten en verhoogde druk. Dankzij Robotic Proces Automation (RPA) kunnen steeds meer taken die tot op heden door mensen worden uitgevoerd, betrouwbaar aan machines worden uitbesteed.\\
Op de markt neemt het aantal RPA-providers snel toe, waardoor het niet makkelijk is om de juiste partner te selecteren. Eens te meer omdat de technologie zodanig nieuw is, gaat men vrijwel blind shoppen.\\
In dit onderzoek richt ik me op de belangrijkste criteria bij
de keuze van het RPA-platform: de kostprijs, de community, de AI-capaciteiten en de mate waarin custom integraties mogelijk zijn. Met verschillende platformen heb ik eenzelfde proces vergelijkbaar geautomatiseerd. Op die manier kon ik, na communicatie met de leveranciers en een marktonderzoek, het RPA-landschap in kaart brengen.

%----------------------------------------------------------------------------------------
%	GEOLOGY
%----------------------------------------------------------------------------------------
\color{Black}
\color{HoGentAccent1} 
\section*{Experiment}
\color{black}
We zijn gestart van een exemplarisch bedrijfsproces, waarbij vooropgestelde activiteiten de criteria bepaalden om tot een selectie van RPA-platformen te komen. De provider moest in staat zijn om (1) te werken met het bestandssysteem van de host, (2) een e-mail te versturen en (3) een zelfgeschreven integratie te kunnen implementeren. Dit voorbeeldproces heb ik vervolgens op een set daarvoor geschikte RPA-platformen uitgewerkt en extensief getest op vergelijkbare parameters. Verschillen tussen gratis en premium versies, community support en kostprijs heb ik eveneens geëvalueerd om elke provider te quoteren.

%----------------------------------------------------------------------------------------
%	CONCLUSION
%----------------------------------------------------------------------------------------
\color{HoGentAccent1} 
\section*{Conclusies}
\color{black}
Alle geteste platformen bieden grosso modo dezelfde functionaliteit. Dit bevestigt nog maar eens hoe moeilijk het is om als leek de meest geschikte provider te kiezen! RPA is daarenboven een dure investering. Om je ROI te maximaliseren, is het van cruciaal belang om hier geen fouten te maken.\\
In een strijd om de RPA-markt te domineren, staan drie providers afgetekend aan de leiding: UiPath, Automation Anywhere en BluePrism. BluePrism werd niet mee onderzocht. IntelliBot toonde dat ook een kleine speler zich hiertussen opmerkelijk waardevol kan positioneren. Ze contrasteren fel met platformen als WorkFusion en Microsoft Flow, die ondanks hun ingeburgerde naam toch eerder zwak scoren.\\
UiPath heeft de grootste community en wordt het meest gebruikt, waardoor deze favoriet is ten opzichte van de andere. Hiermee vergeleken, is Automation Anywhere een stuk beperkter wat betreft het bot management platform.

%----------------------------------------------------------------------------------------
%	FORTHCOMING RESEARCH
%----------------------------------------------------------------------------------------
\color{HoGentAccent1} 
\section*{Toekomstig onderzoek}
\color{black}
Omwille van de restrictieve criteria binnen het voorbeeldproces, hebben we heel wat RPA-providers niet onderzocht. Bovendien evolueert de sector snel en komen nieuwe spelers op de markt soms met game-changing oplossingen. Vervolgonderzoek naar andere RPA-platformen kan interessante informatie opleveren over de toekomst en de richting van deze innovatieve technologie.

%----------------------------------------------------------------------------------------

\end{multicols}
\end{document}