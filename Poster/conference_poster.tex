%\title{LaTeX Portrait Poster Template}
%%%%%%%%%%%%%%%%%%%%%%%%%%%%%%%%%%%%%%%%%
% a0poster Portrait Poster
% LaTeX Template
% Version 1.0 (22/06/13)
%
% The a0poster class was created by:
% Gerlinde Kettl and Matthias Weiser (tex@kettl.de)
% 
% Adapter by Jens Buysse for Hogeschool Gent
% This template has been downloaded from:
% http://www.LaTeXTemplates.com
%
% License:
% CC BY-NC-SA 3.0 (http://creativecommons.org/licenses/by-nc-sa/3.0/)
%
%%%%%%%%%%%%%%%%%%%%%%%%%%%%%%%%%%%%%%%%%

%----------------------------------------------------------------------------------------
%	PACKAGES AND OTHER DOCUMENT CONFIGURATIONS
%----------------------------------------------------------------------------------------
\documentclass[a0,portrait]{a0poster}
\usepackage{multicol}
\columnsep=100pt
\columnseprule=3pt
\usepackage[svgnames]{xcolor}
\usepackage{times}
\usepackage{graphicx}
\graphicspath{{figures/}}
\usepackage{booktabs}
\usepackage[font=small,labelfont=bf]{caption}
\usepackage{amsfonts, amsmath, amsthm, amssymb}
\usepackage{wrapfig}
\usepackage[export]{adjustbox}

\begin{document}

%----------------------------------------------------------------------------------------
%	POSTER HEADER 
%----------------------------------------------------------------------------------------

\begin{minipage}[t]{0.75\linewidth}
\VeryHuge \color{HoGentAccent1} \textbf{RPA, automatisatie van morgen, vandaag} \color{Black}\\
\huge \textbf{Pessemier Mout, Lavaert Laurens, Decorte Johan}\\[0.5cm]
\huge Hogeschool Gent, Valentin Vaerwyckweg 1, 9000 Gent\\[0.4cm]
\Large \texttt{mout.pessemier@hogent.be} \\
\end{minipage}
%
\begin{minipage}[t]{0.25\linewidth}
\includegraphics[width=13cm,right]{figures/HOGENT_Logo_Pos_rgb.png} 

\end{minipage}

\vspace{1cm}

%----------------------------------------------------------------------------------------

\begin{multicols}{2}

%----------------------------------------------------------------------------------------
%	ABSTRACT
%----------------------------------------------------------------------------------------

\color{HoGentAccent1}

\begin{abstract}
	Robotic Process Automation is een sterk opkomende sector die voor vele bedrijven de oplossing kan zijn om processen efficiënt te gaan automatiseren. Het neemt een repetitieve taak weg van een werknemer die zich nu kan focussen op belangrijkere taken. Hierdoor verminderd het aantal fouten die gemaakt worden en kan het bedrijf meer werk verzetten.
\end{abstract}

%----------------------------------------------------------------------------------------
%	INTRODUCTION
%----------------------------------------------------------------------------------------
\color{HoGentAccent1} 
\section*{Introductie}
\color{black}
In de bedrijfswereld wordt een sterke nadruk gelegd op het efficiënt uitvoeren van processen. Een manier om de efficiëntie te verhogen is het automatiseren van deze bedrijfsprocessen. Robotic Process Automation (RPA) is een manier om deze automatisering te verwezenlijken. Nu rest enkel nog de vraag, welk platform? In de zee van RPA-providers is het niet gemakkelijk om de juiste provider te kiezen die past bij uw bedrijf. Er moet onder andere rekening gehouden worden met de prijs, de community, de mogelijkheden om eigen integraties toe te voegen en AI capaciteiten.\\
Tijdens dit onderzoek zijn deze eigenschappen en nog enkele anderen onderzocht geweest om, de keuze voor een bepaald platform te kiezen, makkelijker te maken. Dit wordt gefaciliteerd door onder andere het uitwerken van een proces op de verschillende platformen, de communicatie met de verschillende bedrijven achter de platformen en een marktonderzoek.

%----------------------------------------------------------------------------------------
%	GEOLOGY
%----------------------------------------------------------------------------------------
\color{Black}
\color{HoGentAccent1} 
\section*{Experimenten}
\color{black}
Het uitgevoerde experiment bestaat uit een voorbeeldproces te gaan automatiseren op de verschillende onderzochte RPA-platformen. Dit proces bestaat uit een aantal vooropgestelde activiteiten waaraan verwacht werd dat de provider minstens kon voldoen. Zo werd de nadruk gelegd op het werken met het bestandssysteem van de host, het versturen van een mail en het voorzien van een zelfgeschreven integratie.\\
Voor elk platform is het proces uitgewerkt geweest en is de eigen activiteit geïntegreerd. Nadien is er extensief getest geweest. Ook is er onderzoek gedaan naar het verschil tussen de gratis versie en de premium versie, het forum, de community en de prijs. Op de gevonden bevindingen zijn nadien scores aan elke provider toegekend.

%----------------------------------------------------------------------------------------
%	CONCLUSION
%----------------------------------------------------------------------------------------
\color{HoGentAccent1} 
\section*{Conclusies}
\color{black}
De verschillende platformen omvatten ruw gezien dezelfde functionaliteit. Dit bevestigd nog maar eens de moeilijkheid om de juiste provider te kiezen. Wat wel uit het experiment naar boven komt is dat RPA duur is. Ook wordt de race tussen de 'Big Three' duidelijk. Voor de onderzochte providers, komt UiPath als duidelijke winnaar uit de bus. Het platform heeft de grootste community en wordt het meest gebruikt. Het marktonderzoek bevestigd dan ook deze bevindingen. Op de tweede plek komt Automation Anywhere. [...] Daarnaast kan er besloten worden dat niet alle kleine providers het slecht doen. Daarentegen is er de andere kant van de munt die het minder mooi verhaal beschrijft over WorkFusion die nog een lange weg te gaan heeft voor ze ooit kunnen concurreren. Bij Microsoft Flow kan de kritiek rond de werkwijze en mogelijkheden van het platform niet achter gelaten worden.

%----------------------------------------------------------------------------------------
%	FORTHCOMING RESEARCH
%----------------------------------------------------------------------------------------
\color{HoGentAccent1} 
\section*{Toekomstig onderzoek}
\color{black}
Uit de zee van providers zijn er slechts enkele uitgekozen die onderzocht geweest zijn. Verder onderzoek naar de andere RPA-providers kan nieuwe kandidaten naar boven brengen om te concurreren met de 'Big Three'. Ook lebert verder onderzoek naar de RPA markt interessante informatie op over de toekomst en richting dat deze technologie zal inslaan.

%----------------------------------------------------------------------------------------

\end{multicols}
\end{document}