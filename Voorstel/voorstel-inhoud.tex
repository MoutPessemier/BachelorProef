%---------- Inleiding ---------------------------------------------------------

\section{Introductie}
\label{sec:introductie}
De hoeveelheid tijd die verloren gaat aan het uitvoeren van repetitieve, dagelijkse taken is een optimalisatieprobleem dat de meeste bedrijven ervaren. \\
Een manier om dit probleem aan te pakken is de technologie, Robotic Process Automation (RPA). Doordat RPA processen en systemen gaat automatiseren, kunnen de werknemers zich focussen op belangrijkere en productievere taken.\\
 Om uit te zoeken welke voor- en nadelen RPA met zich meebrengt, bestaat het onderzoek uit volgende vragen:
\begin{itemize}
	\item Welke recente ontwikkelingen zijn relevant binnen het RPA segment? 
	\item Waar kan RPA een meerwaarde bieden, is dit interessant voor elk bedrijf?
	\item Welke kennis en vaardigheden vergt het om een RPA proces op te zetten?
	\item Hoe betrouwbaar is RPA, hoe zit het met fout tolerantie?
	\item Wat als de applicatie vast loopt?
\end{itemize}

%---------- Stand van zaken ---------------------------------------------------

\section{Literatuurstudie}
\label{sec:literatuurstudie}
RPA, is een toepassing van software en gestructureerde inputs, gericht op het automatiseren van business processen. Gebruik makend van RPA tools, kan een bedrijf software configureren voor het vastleggen en interpreteren van applicaties om o.a. transactionele processen, data manipulatie en het communiceren met andere digitale systemen uit te voeren. Deze automatiseringen kunnen gaan van een simpele auto-reply mail tot het inzetten van duizenden bots of het werken binnen een Enterprise Resource Planning (ERP) systeem. \autocite{whatIsRPA} \\
Het gebruik van de technologie stijgt sterk, met een groeipercentage van 57\%. Wereldwijd is in 2018, \$680 miljoen gespendeerd aan RPA systemen en men verwacht tegen 2022 zelfs  \$2.4 miljard.  \autocite{isRPAWorthIt}  Dit is ook niet onlogisch aangezien een goed geïmplementeerd RPA proces gemakkelijk 5 tot 10 mensen vervangt. \autocite{rpaMomentum}

\subsection{Voordelen van RPA}
Door RPA zal de loonkost verminderen en de gemaakte fouten dalen aangezien RPA slecht 1 juist voorbeeld nodig heeft om het geautomatiseerd proces correct uit te voeren.\\
Het voornaamste voordeel is het feit dat RPA aan bedrijven de mogelijkheid geeft om processen en systemen die niet gemaakt zijn om te automatiseren, toch geautomatiseerd te krijgen. Hieronder vallen ook de systemen die sterk verouderd zijn en/of  traag zijn. \\
Bots zijn over het algemeen gemakkelijk te implementeren en goedkoop. Ze hebben geen extra zelfgeschreven software of diep geïntegreerde systemen nodig.\\
Deze automatiseringen kunnen extra ondersteund worden door het gebruik van Machine Learning  (ML),  Natural Language Processing (NLP) en speach recognition tot  Intelligent Automation (IA) waarbij RPA zich ook kan aanpassen in een licht veranderende omgeving. \autocite{whatIsRPA}

\subsection{Valkuilen van RPA}
Ondanks de vele voordelen zijn er ook enkele gevaren aan RPA verbonden. Met een automatiserende technologie bestaat het risico dat jobs verdwijnen. Zo'n technologie vloeit meestal voort uit een industriële revolutie. Sedert de eerste is het verdwijnen van banen een probleem waarbij elke nieuwe revolutie er meer en meer banen wegvallen. RPA situeert zich in de $4^{de}$ industriële revolutie. De meeste bedrijven proberen de te vervangen werknemers in te zetten voor nieuwe jobs maar toch wordt er geschat dat zeker 9\% vervangen zal worden door RPA systemen.\\
Het installeren van duizenden RPA systemen blijkt ingewikkelder en duurder dan verwacht. Slecht 3\% van de bedrijven heeft in 2018 meer dan 50 RPA software packets succesvol kunnen opzetten. Ook verandert het platform waarmee de bots interageren op regelmatige basis waardoor de nodige flexibiliteit tot aanpassen niet standaard aanwezig is.\\
Een laatste struikelblok is dat een kleine wijziging aan een formulier ervoor  kan zorgen dat het geleverde werk volledig verloren gaat omdat RPA niet meer weet wat te doen met deze nieuwe data. \autocite{whatIsRPA}\\
\newline
Om te bepalen of een proces kan vervangen worden door RPA zijn vier regels vooropgesteld:
\begin{itemize}
	\item Het proces moet gebaseerd zijn op duidelijke regels
	\item Het proces moet regelmatig herhaald worden of een eenduidige activator hebben
	\item Het proces moet meerdere in- en outputs hebben
	\item Het proces moet een voldoende groot volume hebben
\end{itemize} \autocite{explainRPA}

\subsection{Automatic Data Processing}
Data processing (DP) is het omzetten van verworven data in relevante informatie, meestal uitgevoerd door een data analist. \autocite{whatIsDP}

\subsubsection{De 6 stappen van data verwerking}
\begin{itemize}
	\item Data collectie: verzamelen van data
	\item Data voorbereiding: het 'opkuisen' van de data
	\item Data input: de data in het systeem inbrengen
	\item Data verwerken
	\item Data interpretatie: de resultaten van de verwerking interpreteren
	\item Data opslag: de data opslaan voor verder gebruik
\end{itemize}  \autocite{whatIsDP}

Automatic Data Processing (ADP) beschrijft het automatisch verwerken van inkomende data. Deze stappen volgen in het algemeen eenzelfde stappenplan waardoor deze geautomatiseerd kunnen worden. \\
Aangezien de toekomst van data processing in 'the cloud' ligt en gebruik gemaakt wordt van elektronische methoden om zijn snelheid en effectiviteit te verhogen \autocite{whatIsDP}, kan aan de hand van RPA deze snelheid nog omhoog en het aantal fouten nog sterk naar beneden gehaald worden. Dit resulteert in bruikbare data van hoge kwaliteit.

%---------- Methodologie ------------------------------------------------------
\section{Methodologie}
\label{sec:methodologie}
Enkele interessante scenario's zullen tijdens het onderzoek praktisch uitgevoerd worden op de labeling tool van Faktion a.d.h.v. een RPA software pakket. Hierbij wordt gekeken 1) hoeveel tijd en kennis nodig is om deze uit te werken, 2) welke bedrijfssectoren voor deze technologie het meest in aanmerking komen 3) of een extra proces makkelijk te implementeren is en 4) welke systemen meerdere RPA processen nodig hebben. \\
Deze scenario's zullen doorlopen worden van makkelijk naar moeilijk.

%---------- Verwachte resultaten ----------------------------------------------
\section{Verwachte resultaten}
\label{sec:verwachte_resultaten}
RPA zal een duidelijke verhoging van de efficiëntie met zich meebrengen doordat medewerkers zich kunnen focussen op het uitvoeren van productievere taken en dit niet alleen binnen de IT sector. RPA's kunnen namelijk veel systemen en processen automatiseren. Er wordt vanuit gegaan dat sectoren zoals sales, boekhouding en management hier ook baat bij hebben. De opzet van een scenario zal binnen een week succesvol gebeuren en het onderhoud van dit process zal zonder verdere problemen verlopen. Ook het implementeren van nieuwe processen gebeurt probleemloos. Er wordt geschat dat het foutgehalte door automatiseren sterk vermindert t.o.v. wanneer werknemers deze taken uitvoeren.

%---------- Verwachte conclusies ----------------------------------------------
\section{Verwachte conclusies}
\label{sec:verwachte_conclusies}
RPA processen zijn het investeren waard om een groot aantal geïmplementeerde processen te ondersteunen. Deze geautomatiseerde processen leveren veel tijdswinst op en kunnen in meerdere sectoren geïmplementeerd worden. De productiviteit stijgt zelfs bij systemen die oorspronkelijk niet voorzien zijn om te automatiseren.  De foutenmarge verkleint waardoor de kwaliteit van de onderneming stijgt.

