%%=============================================================================
%% Methodologie
%%=============================================================================

\chapter{\IfLanguageName{dutch}{Methodologie}{Methodology}}
\label{ch:methodologie}

%% TODO: Hoe ben je te werk gegaan? Verdeel je onderzoek in grote fasen, en
%% licht in elke fase toe welke stappen je gevolgd hebt. Verantwoord waarom je
%% op deze manier te werk gegaan bent. Je moet kunnen aantonen dat je de best
%% mogelijke manier toegepast hebt om een antwoord te vinden op de
%% onderzoeksvraag.

%\lipsum[21-25]

Aan de hand van de RPA provider UiPath zullen enkele concrete processen geautomatiseerd worden op Metamaze, het automated document processing platform bij Faktion zelf.


\section{Voorbereiding}
Ter voorbereiding van het te automatiseren proces bij Faktion heb ik mij eerst ingewerkt op het platform waar dit proces zich voordoet. Ook heb ik de volledige 'RPA Developer Essentials' cursus aangeboden door UiPath gevolgd. Het proces zal dan ook binnen UiPath uitgewerkt worden. Er is gekozen geweest voor UiPath omdat dit een van de grote drie RPA software providers, naast Automation Anywhere en Blue Prism, is en omdat ze een gratis RPA developer cursus aanbieden. Ook omdat UiPath een community edition heeft waarop 3 bots kunnen worden geüpload.

Eens de cursus afgewerkt en het te automatiseren proces uitgelegd was, ben ik begonnen met research te doen rond het schrijven van een connector piece, een tussenstuk om de RPA implementatie van externe bedrijven te kunnen integreren in het proces uitgevoerd op Metamaze.

