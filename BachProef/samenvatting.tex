%%=============================================================================
%% Samenvatting
%%=============================================================================
\chapter*{\IfLanguageName{dutch}{Samenvatting}{Abstract}}
Wie zoekt naar \acrshort{rpa}-aanbieders op het internet, krijgt een lijst van providers. Iedere provider heeft voor- en nadelen. Om hieruit het juiste platform uit te kiezen dat past bij de bedrijfscultuur en mentaliteit is geen gemakkelijke taak. Er zijn vele aspecten die overwogen moeten worden om deze keuze succesvol te maken. Daarom is in deze thesis onderzoek uitgevoerd geweest naar enkele van deze providers om hun sterke en zwakke punten in kaart te brengen om zo de juiste keuze te maken.\\
Het onderzoek gaat na bij vijf providers (UiPath, Automation Anywhere, WorkFusion, IntelliBot en Microsoft Flow) waar ergens ze zich situeren in de markt, waarin ze uitblinken als platform en waaraan nog verbeteringen nodig zijn. Bij de selectie van de vijf aanbieders is rekening gehouden met criteria zoals de grootte van het bedrijf en het soort \acrshort{rpa}. Dit is gerealiseerd geweest door met de verschillende platformen te werken om een demo proces te automatiseren. Daarnaast wordt er ook contact gelegd met het bedrijf en de community achter de provider.\\
Elke aanbieder zijn resultaat staat in detail beschreven in hoofdstuk drie: Methodologie. Kort samengevat, eerst is door de lessen gegaan die aangeboden werden op het platform zelf. Nadien is het proces geautomatiseerd geweest behalve de zelfgeschreven \gls{activiteit}. Daarna is deze \gls{activiteit} geïmplementeerd en geïntegreerd geweest in de \gls{workflow}. Dit is nadien extensief getest geweest op fouten.\\
Uiteindelijk werden aan de vooropgestelde criteria punten toegewezen nadat elk platform getest was. Deze criteria zijn onderverdeeld in drie categorieën: technische-, bedrijfs- en financiële aspecten.\\
De belangrijkste conclusies van dit onderzoek zijn dat UiPath duidelijk als beste naar voor komt. Ook kan besloten worden dat de kleine providers in de toekomst waarschijnlijk uit de markt zullen verdwijnen. Dit is te wijten aan de voorspelling dat het een race zal worden tussen de drie grote providers (UiPath, Automation Anywhere en Blue Prism). Wie van de drie zal de markt uiteindelijk beheersen? Op dit moment ziet het er naar uit dat UiPath zal winnen.\\
Dit onderzoek is alles behalve volledig. Er zijn vele andere providers die nog niet onderzocht geweest zijn en mogelijks ook het vermelden waard zijn. Daarnaast zijn er ook nog andere criteria die onderzocht kunnen worden. Als laatste kan ook een diepgaander onderzoek uitgevoerd worden naar de race tussen de 'Big Three'.