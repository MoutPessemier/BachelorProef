%%=============================================================================
%% Samenvatting
%%=============================================================================
\chapter*{\IfLanguageName{dutch}{Samenvatting}{Abstract}}
Als gezocht wordt naar \acrshort{rpa} aanbieders op het internet, dan wordt een lijst van providers voorzien. Hierbij heeft iedere provider wel voor- en nadelen. Om hier dan het juiste platform uit te kiezen dat past bij de bedrijfscultuur en mentaliteit is geen gemakkelijke taak. Er zijn vele aspecten die overwogen moeten worden om zo een keuze succesvol te maken. Daarom is in deze thesis onderzoek uitgevoerd geweest naar enkele van deze providers met hun sterke en zwakke punten in de hoop deze keuze makkelijker te maken of in de juiste richting te sturen.\\
Het onderzoek gaat na bij 5 providers (UiPath, Automation Anywhere, WorkFusion, IntelliBot en Microsoft Flow) waar ergens ze zich situeren in de markt, waarin ze uitblinken als platform en waaraan nog verbeteringen nodig zijn. Bij de selectie van de 5 aanbieders is rekening gehouden met grootte en soort \acrshort{rpa}. Dit is gerealiseerd geweest door met de verschillende platformen te werken om een demo proces te automatiseren en door contact te leggen met het bedrijf en de community achter de provider.\\
De hele uitwerking van elke aanbieder staat in detail beschreven in hoofdstuk drie: Methodologie. Kort samengevat, eerst is door de lessen gegaan die aangeboden worden op het platform zelf. Nadien is het proces geautomatiseerd geweest behalve de zelf geschreven activiteit. Nadien is deze activiteit geïmplementeerd en geïntegreerd geweest in de workflow. Dit is nadien extensief getest geweest op fouten.\\
Na met elk platform gewerkt te hebben, zijn aan verschillende vooropgestelde criteria punten toe gewezen. Deze criteria zijn onderverdeeld in 3 categorieën: technische-, bedrijfs- en financiële aspecten. Hoe hoger de score van de aanbieder, hoe beter.\\
De belangrijkste conclusies die uit dit onderzoek kunnen getrokken worden zijn dat UiPath ver boven de rest staat en dat de kleine providers in de toekomst waarschijnlijk uit de markt zullen verdwijnen aangezien het een race zal worden tussen de drie grote providers (UiPath, Automation Anywhere en Blue Prism) over wie de markt uiteindelijk zal beheersen. Op dit moment ziet het er naar uit dat UiPath zal winnen.\\
Mogelijke uitbreidingen op dit onderzoek zijn natuurlijk de andere providers die niet onderzocht geweest zijn. Zoals eerder vermeld is er een hele hoop aan providers waar er slechts 5 uit gekozen zijn. Een mogelijks tweede uitbreiding is dat andere criteria ook nog onderzocht kunnen worden.