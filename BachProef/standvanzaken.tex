\chapter{\IfLanguageName{dutch}{Stand van zaken}{State of the art}}
\label{ch:stand-van-zaken}

% Tip: Begin elk hoofdstuk met een paragraaf inleiding die beschrijft hoe
% dit hoofdstuk past binnen het geheel van de bachelorproef. Geef in het
% bijzonder aan wat de link is met het vorige en volgende hoofdstuk.

% Pas na deze inleidende paragraaf komt de eerste sectiehoofding.

%Dit hoofdstuk bevat je literatuurstudie. De inhoud gaat verder op de inleiding, maar zal het onderwerp van de bachelorproef *diepgaand* uitspitten. De bedoeling is dat de lezer na lezing van dit hoofdstuk helemaal op de hoogte is van de huidige stand van zaken (state-of-the-art) in het onderzoeksdomein. Iemand die niet vertrouwd is met het onderwerp, weet nu voldoende om de rest van het verhaal te kunnen volgen, zonder dat die er nog andere informatie moet over opzoeken \autocite{Pollefliet2011}.

%Je verwijst bij elke bewering die je doet, vakterm die je introduceert, enz. naar je bronnen. In \LaTeX{} kan dat met het commando \texttt{$\backslash${textcite\{\}}} of \texttt{$\backslash${autocite\{\}}}. Als argument van het commando geef je de ``sleutel'' van een ``record'' in een bibliografische databank in het Bib\LaTeX{}-formaat (een tekstbestand). Als je expliciet naar de auteur verwijst in de zin, gebruik je \texttt{$\backslash${}textcite\{\}}.
%Soms wil je de auteur niet expliciet vernoemen, dan gebruik je \texttt{$\backslash${}autocite\{\}}. In de volgende paragraaf een voorbeeld van elk.

\acrshort{rpa} is een onderdeel van \acrfull{bpa}, een overkoepelende term gebruikt voor het beschrijven van technologieën die activiteiten en workflows, waaruit business taken bestaan, uitvoeren met zo min mogelijk menselijke tussenkomst. \autocite{everythingRPA}

\acrshort{rpa} kan gecombineerd worden samen met \acrfull{ai} om zo langere en moeilijkere taken op zich te nemen. Hierdoor worden ze door sommigen beschreven als zelf-verbeterende digitale werkkrachten. Enkele velden waarbinnen \acrshort{rpa} kan gecombineerd worden met \acrshort{ai} zijn \acrfull{ocr} en \acrfull{nlg}. \autocite{everythingRPA}

\section{Algemene informatie}

\subsection{Waarvoor kan \acrshort{rpa} gebruikt worden}
Enkele taken waarvoor \acrshort{rpa} kan worden gebruikt:
\begin{itemize}
	\item data overzetten van de ene applicatie naar de andere
	\item processen automatiseren
	\item website scraping
	\item data verzamelen en analyseren
	\item reporting
	\item emails verzenden
	\item werken met spreadsheets, PDFs, texteditors
\end{itemize}
\autocite{everythingRPA} \autocite{idrRPA}

\subsection{In welke sectoren wordt \acrshort{rpa} gebruikt}
Onderdelen van een onderneming waarbinnen deze taken kunnen gebruikt worden:
\begin{itemize}
	\item Klantenservice: validatie van handtekeningen, uploaden van ingescande documenten, valideren van informatie voor automatische aanvaarding of afwijzing
	\item Accounting: procedure to pay, accounting, belastingen
	\item \acrfull{hr}: payroll, time management, recruitment
	\item IT management en services: source-code control management, incident management, optimaliseren van e-mail notificaties
	\item Supply chain management: order processing, payment processing, monitoring van de stock
\end{itemize}
\autocite{everythingRPA}

\subsection{Welke processen kunnen geautomatiseerd worden}
\acrshort{rpa} is kan niet gebruikt worden om zo maar eender welk business proces te gaan automatiseren. Een proces toont best een aantal karakteristieken die het geschikt maken om te automatiseren:
\begin{itemize}
	\item repetitief
	\item gebaseerd op gestructureerde, digitale data
	\item duidelijk afgebakende regels met weinig tot geen uitzonderingen
	\item error-prone als het uitgevoerd wordt door een werknemer
	\item tijdsgebonden
\end{itemize}

\subsection{Automation-First gedachtegang}
In de Automation-First gedachtegang moet je elk bedrijfsproces kritisch bekijken met de gedacht of het niet automatiseerbaar is. Deze gedachtegang overnemen en toepassen binnen het eigen bedrijf is de eerste stap op de weg naar het maken van de digitale transformatie binnen het bedrijf. Werken via deze Automation-First mindset zorgt ervoor dat bedrijven zich sneller en efficiënter kunnen bewegen. Het neemt alledaagse lasten weg van de werknemers om zo meer tijd vrij te maken voor het oplossen van complexe en uitdagende problemen.

Het overnemen van een automation-first mindset is de eerste stap naar het inzetten van een digitale transformatie in een onderneming. Denken op die manier staat toe aan een organisatie om sneller en meer efficiënt te bewegen binnen de sector door onder andere te helpen met het verhogen van de klantenservice of beter te kunnen inspelen op marktwijzigingen en meer efficiënt te kunnen werken.

\section{Verschillen met soortgelijke technologieën}

\subsection{Gewone automatisatie}
Het grote verschil tussen \acrshort{rpa} en gewone automatisatie is de kennis van coderen die nodig is om het doel te bereiken. Bij gewone automatisatie schrijft een programmeur code die een bepaalde taak automatiseert. Bij \acrshort{rpa} daarentegen wordt over het algemeen weinig tot geen code gebruikt om een proces te gaan automatiseren.

\subsection{Business Process Management}
Ook is er een verschil tussen \acrshort{rpa} en \acrfull{bpm}. Beide willen het aantal fouten verminderen en efficiëntie verhogen maar ze doen dat op faliekant andere manieren. \acrshort{bpm} gaat helderheid scheppen bij business processen door deze te gaan modelleren en in kaart te brengen waar \acrshort{rpa} deze processen juist wil gaan automatiseren, daarom niet in kaart brengen.

\subsection{Intelligent Process Automation}
Als laatste is er \acrfull{ipa} tegenover \acrshort{rpa}. Er is hier minder sprake van een verschil aangezien \acrshort{ipa} \acrshort{rpa} combineert met \acrshort{bpm}, \acrshort{ai} en \acrfull{ml} om processen optimaal te gaan verbeteren.
\autocite{everythingRPA}

\section{De verschillende \acrshort{rpa} Architecturen}
Er zijn verschillende soorten \acrshort{rpa} die kunnen gebruikt worden binnen een bedrijfscontext:

\begin{itemize}
	\item Assisted (Attended) automation: \acrshort{rpa} loopt op een gebruiker zijn desktop om hem te helpen een proces op een snellere manier kunnen af te werken. Dit leidt in het algemeen tot een vermindering van kosten en een betere klantenservice. Een groot nadeel aan deze architectuur is dat een wijziging of onregelmatigheid in de beeldscherm instellingen zoals het wijzigen van display instellingen als lettergrootte kan er voor zorgen dat \acrshort{rpa} faalt in het uitvoeren van de vooropgestelde taak.
	\item Unassisted (Unattended) automation: heeft geen menselijke interactie nodig, \acrshort{rpa} draait het systeem zelf en laat pas iets weten aan de werknemer wanneer iets fout gaat en hij het zelf niet kan oplossen. Dit zijn de 24/7 digitale werkkrachten die geassocieerd worden met \acrshort{rpa}.
	\item Hybrid \acrshort{rpa}: Een combinatie van vorige 2 architecturen waarbij de werknemer en de \acrshort{rpa} robot als een team samen werken, onderling taken van en naar elkaar sturen. De hybride methode overkoepelt op deze manier zowel het werk dat alleen uitgevoerd kan worden door de robot als ook het werk waarvoor menselijke interactie nodig is.
\end{itemize}

Daarnaast kan \acrshort{rpa} op verschillende manieren geïmplementeerd worden, zo kan de low-code development mentaliteit gevolgd worden waarbij amper iets van kennis nodig is. Dit is perfect voor bedrijven met een beperkte IT kennis. Aan de andere kant kan ook veel gecodeerd worden om deze robots op te zetten.
\autocite{everythingRPA}

\section{Voordelen van \acrshort{rpa}}

\subsection{Efficiëntie}
Vanwaar komt de efficiëntie van \acrshort{rpa}?
\begin{itemize}
	\item Robots werken 24/7: in tegenstelling tot werknemers blijven robots werken na de werkuren. Ze worden ook niet moe en moeten nooit naar de wc. Ze kunnen ook hun concentratie niet verliezen. Als al deze punten opgeteld worden, dan kan slechts 1 conclusie getrokken worden: \acrshort{rpa} robots zijn harde werkers die nooit pauze nodig hebben.
	\item Snellere uitvoertijd: het aantal transacties die een robot kan afwerken per uur is veel meer dan het aantal dat een werknemer ooit zou kunnen afwerken binnen hetzelfde tijdskader.
	\item Het vinden van fouten in het systeem: \acrshort{rpa} verzamelt informatie over het systeem waar het op werkt en giet dit in een analytisch overzicht. Hier kan nadien informatie uit gehaald worden waar een vertraging of opstopping in het proces zich voordoet.
	\item Meer gemotiveerde werknemers: waar \acrshort{rpa} het proces zal verbeteren, zal ook de werknemer die verlost is van deze lastige taak meer motivatie tonen naar de leuke processen waaraan deze nu kan werken.
	\item Doe meer met minder: Waar \acrshort{rpa} kan gezien worden als een manier om even veel werk te verrichten met minder werknemers is het beter om te kijken naar de mogelijkheden die \acrshort{rpa} biedt om meer werk te verrichten met hetzelfde aantal werknemers wat tot een groei van de onderneming kan leiden.
\end{itemize}

\autocite{efficiencyRPA}
\subsection{Nauwkeurigheid}
Als een proces juist opgezet is, dan kunnen \acrshort{rpa} robots dit proces blijven juist uitvoeren zonder vermoeid te raken, hierdoor daalt het foutpercentage drastisch tot dicht bij 0\%. \autocite{efficiencyRPA}

\subsection{Kostvermindering}
\acrshort{rpa} zorgt ervoor dat de hoeveelheid werk die afgewerkt geraakt zal stijgen met eenzelfde aantal werknemers of staat toe aan een bedrijf om met minder werknemers even veel werk klaar te krijgen. \autocite{efficiencyRPA}

\subsection{Werknemer productiviteit en moraal}
Door werknemers van de saaie en repetitieve taken weg te nemen en ze te laten werken aan taken die wel degelijk belang hebben binnen een bedrijf stijgt de inzet, moraal en productiviteit die van een werknemer kan verwacht worden. \autocite{efficiencyRPA}

\subsection{Digitale transformatie}
Door het gebruik van \acrshort{rpa} kan een bedrijf delen of hele processen gaan automatiseren zonder wijzigingen aan te brengen aan legacy-code of te investeren in integratie van back-end systemen met reeds bestaande systemen. Hierdoor kunnen ze verder door gaan in deze digitale transformatie. \autocite{efficiencyRPA}

\section{Nadelen van \acrshort{rpa}}
Gemiddeld falen 30 tot 50\% de eerst \acrshort{rpa} projecten. \autocite{everythingRPA} Een van de grootste redenen waarom zo veel projecten falen is het feit dat men denkt dat een \acrshort{rpa} project een puur IT project is. Dit is niet correct, het \acrshort{rpa} proces moet opgezet worden door werknemers die kennis hebben van \acrshort{rpa} maar ook de werknemers die het proces snappen en de 'as-is' kunnen interpreteren om zo deze werknemers te sturen in de juiste richting.

Enkele andere veelvoorkomende risico's zijn:

\subsection{Onderschatten van human capital}
Ondanks net gezegd dat \acrshort{rpa} niet alleen IT is, is de aanwezigheid van IT van vitaal belang om een \acrshort{rpa} project te kunnen blijven ondersteunen/onderhouden. Iedereen kan een proces automeren maar het draaiende houden en onderhouden is al een pak technischer en dus niet voor iedereen weggelegd. \autocite{everythingRPA}

\subsection{Security risico's}
\acrshort{rpa} brengt een aantal security risico's met zich mee. Zo kan een robot die niet goed geïmplementeerd is bijvoorbeeld frauduleuze zaken uitvoeren of toegang krijgen tot en werken met sensitieve data en beheerders privileges gaan misbruiken. Het is dus belangrijk om de processen zorgvuldig en aandachtig te gaan opstellen en zeker te controleren dat deze risico's vermeden kunnen worden. \autocite{everythingRPA} \autocite{predictionRPA}

\section{Moeilijkheden van \acrshort{rpa}}
\acrshort{rpa} moet gebruikt worden om de juiste processen te gaan automatiseren, maar wat zijn nu de juiste processen? Bedrijven lijken het zeer moeilijk te vinden om die processen aan te duiden die werkelijk baat hebben moesten ze geautomatiseerd worden. Hierbij komt meestal nog eens kijken dat business de bots implementeren zonder echt te snappen wat ze doen of dat IT de processen automatiseert zonder deze echt te snappen. Dit leidt tot het repliceren van inefficiëntie maar dan op snelheid.  \autocite{cFutRPA}

Een manier om dit op te lossen is door gebruik te maken van process mining. Deze technologie maakt gebruik van business data en gebruiker interactie data om een duidelijk beeld te geven wat en hoe processen nu werkelijk in elkaar zitten en gebruikt worden. Elke stap die genomen wordt tijdens het proces wordt vastgelegd en gebruikt om dit beeld verder op te bouwen. Dit zorgt voor een grotere verstandhouding rond het proces en zorgt ervoor dat bedrijven de wrijving binnen de loop van een proces kunnen identificeren en verbeteren. \autocite{cFutRPA}

\subsection{Meest gemaakte fouten}
Naast het succes blijkt het implementeren van \acrshort{rpa} toch niet zo eenvoudig. Zo faalde meer dan 40\% van de projecten op een van volgende 4 punten:
\begin{itemize}
	\item implementatie tijd
	\item implementatie kost
	\item het verminderen van kosten door \acrshort{rpa}
	\item voordeel bij het uitvoeren van analyses
\end{itemize}

Er zijn heel wat verschillende zaken die aan de oorzaak van het falen van een \acrshort{rpa} project kunnen liggen. \autocite{pitfallsRPA}

\subsubsection{Fouten op organisatie niveau}
Op eenzelfde lijn zitten, eenzelfde gedachte gang binnen een organisatie is een sleutel aspect tot succes. Als dit niet het geval is, treden enkele veelvoorkomende fouten op. Zo zijn de grootse fouten op organisatie niveau het niet genoeg ondersteunen van het \acrshort{rpa} project. Hierdoor is er te weinig IT ondersteuning om het proces te implementeren of het leiderschap dat niet achter het project staat. Ook is er kans dat het team te weinig tijd in het project steekt. Het is van uitermate belang dat iedereen eenzelfde visie deelt over het project en dat iedereen binnen de organisatie er achter staat. \autocite{pitfallsRPA}

\subsubsection{Fouten op proces niveau}
De meest belangrijke beslissing is welk proces nu zal geautomatiseerd worden. Hier zitten dan ook de grootste fouten op proces niveau. Zo wordt er vaak gekozen om processen te automatiseren die zeer frequent wijzigen of weinig tot geen business waarde opleveren. Ook wordt geregeld een proces gekozen dat cognitieve beslissingen nodig heeft om te werken of te complexe processen bevat. \acrshort{rpa} biedt geen oplossing voor dit soort problemen. Hiervoor moet al gekeken worden naar \acrshort{rpa} 2.0 of \acrshort{ipa}. \autocite{pitfallsRPA}

\subsubsection{Fouten op implementatie niveau}
Op implementatie niveau zit de grootste fout dat men alles intern wil houden. Een eigen team die een eigen oplossing zoekt voor een (veelvoorkomend) proces. Het is niet verkeerd om beroep te doen op \acrshort{rpa} experts die kunnen helpen met het opzetten van agenten. \autocite{pitfallsRPA}

Daarnaast wordt geprobeerd om elk proces voor 100\% te automatiseren. Dit kan nadelig uitvallen aangezien 80\% makkelijk te automatiseren is maar die laatste 20\% juist zeer moeilijk en duur. In deze gevallen is het beter om die laatste 20\% toch te laten uitvoeren door een werknemer. \autocite{pitfallsRPA}

\subsubsection{Fouten op technisch niveau}
De meest voorkomende fouten op technisch niveau gaan over het kiezen van te programmeer intensieve oplossingen die niet schaalbaar zijn. In plaats hiervan kan gekeken worden naar \acrshort{rpa} marketplaces. \autocite{pitfallsRPA}

\acrshort{rpa} is een evoluerende sector. Het is belangrijk dat de laatste trends gevolgd worden en processen geoptimaliseerd blijven door nieuwe technieken. Hier schuilt het gevaar dat oplossingen gekocht worden op \acrshort{rpa} marktplaatsen om tijd uit te sparen. Nadien blijkt dan dat de gekochte oplossing outdated is en helemaal niet meer zo efficiënt. Hierdoor is de onderneming zowel tijd als geld kwijt aan een suboptimale oplossing. \autocite{pitfallsRPA}

\subsubsection{Fouten na het implementeren}
Het verhaal van \acrshort{rpa} stopt niet eens de agent online is en het werk uitvoert. Ook onderhoud en schaalbaarheid zijn van belang. Men ziet dan ook dat er velen de \acrshort{rpa} service links laten liggen eens deze werkt wat resulteert in extra kosten maar ook een vertraging van de digitale transformatie van een organisatie. \autocite{pitfallsRPA}

\section{De toekomst van \acrshort{rpa}}
We hebben natuurlijk geen glazen bol en kunnen dus niets met zekerheid zeggen over de toekomst van \acrshort{rpa}. Wel kan gespeculeerd worden hoe deze technologie zal evolueren in de komende jaren. Experten zijn het er over het algemeen mee eens dat \acrshort{rpa} de toekomst is van automatisering binnen IT. \autocite{futRPA}

\subsection{Enkele nummers}
\acrshort{rpa} is ver gekomen sinds 2016. Toen lag het bedrag van gekochte \acrshort{rpa} software rond \$73 miljoen. Dit is gestegen tot \$113 miljoen in 2017, \$153 miljoen in 2018 en \$192 miljoen in 2019. Er wordt geschat dat dit bedrag zal stijgen tot \$232 miljoen in 2020 en tegen 2021 zou dit bedrag liggen rond de \$272 miljoen. Hieruit valt duidelijk af te leiden dat er een sterke stijging in het gebruik van \acrshort{rpa} voorlopig aan de gang is en dat dit niet direct blijkt te stoppen. Meer en meer ziet men dat \acrshort{rpa} in de globale markt opgenomen wordt. \autocite{futRPA} Ondanks dat de enorme groei in \acrshort{rpa} uitgaven blijft stijgen, zijn er toch enkele vragen die op komen rond de levensduur van \acrshort{rpa}-markten. \autocite{everythingRPA}

Enkele verwachtingen:
\begin{itemize}
	\item Meer gebruik in de komende jaren door het leggen van de focus op de no-code/low-code oplossingen 
	\item combinatie met \acrshort{ai}, cloud services en \acrshort{ml} om \acrshort{rpa} te verheven tot \acrshort{ipa}
	\item combinatie met andere tools
	\item \acrshort{rpa} marktplaatsen: geen enkele \acrshort{rpa} software aanbieder kan alle in gebruikte processen ondersteunen. Door het oprichten van \acrshort{rpa} marktplaatsen kunnen software bedrijven zelf extensies en plugins schrijven en deze verkopen
\end{itemize} \autocite{futureRPA}\autocite{everythingRPA}

\subsection{\acrshort{rpa} 2.0}
\acrshort{rpa} 2.0, ook wel \acrfull{idr} genoemd, gaat de huidige \acrshort{rpa} gaan verbeteren door gebruik te maken van  \acrshort{ai} en  \acrshort{ml} om de robots slim te maken. Hierdoor moeten ze niet blindelings de regeltjes volgen maar kunnen ze, gebaseerd op de algoritmen en de interpretatie van de data, eigen beslissingen maken. Dit is een duidelijke stap naar het befaamde  \acrshort{ipa}. \autocite{idrRPA}

Vooral deep-learning en reinforcement learning technieken worden gebruikt om \acrshort{rpa} 2.0 te verwezenlijken. \autocite{idrRPA}

\subsection{Hyperautomation}
Hyperautomation gaat \acrshort{rpa} combineren met  \acrshort{ml},  \acrshort{ai} en andere disruptieve technologieën om een end-to-end automatisatie oplossing te vinden die meer impactvol is dan traditionele \acrshort{rpa} of automatisatie. Deze verzameling van technologieën speelt dan ook perfect in op de automation-first mindset die een goedde \acrshort{rpa} ontwikkelaar nodig heeft om de juiste processen te kunnen automatiseren met succes. \autocite{hyperautomation}

\subsection{IPA}
Intelligent Process Automation gaat \acrshort{rpa} gaan combineren met onder andere  \acrshort{ai} en  \acrshort{ml} technieken om aan de software agent toe te staan dat hij complexere processen kan gaan automatiseren waarbij enige vorm van beslissingen maken nodig is.

IPA omvat 5 technologieën:
\begin{itemize}
	\item \acrshort{rpa}: een software automatisatie tool die routinematige taken zoals data extractie en opschoning automatiseren door een grafische gebruikers interface.
	\item Smart workflow: een process-management software tool die taken uitgevoerd door zowel man als machine integreert.
	\item  \acrshort{ml}: algoritmen die patronen in herkennen in data zoals dagelijkse performantie data. Dit gebeurd aan de hand van gesuperviseerd of ongesuperviseerd leren. Bij gesuperviseerd leren is de data gelabeld en wordt deze data gebruikt om voorspellingen te doen over nieuwe data. Bij ongesuperviseerd leren is de data niet gelabeld. Deze algoritmen worden onder andere gebruikt bij het herkennen van clusters in de data.
	\item  \acrshort{nlg}: software engines die vlotte interacties tussen mensen en technologie maken door het volgen van regels om observaties van data te vertalen.
	\item Cognitive agents: technologieën die  \acrshort{ml} en  \acrshort{nlg} combineren om compleet virtuele werkkrachten of agenten te maken die in staat zijn om taken uit te voeren, communiceren, leren van datasets en zelfs beslissingen maken gebaseerd op de interpretatie van emoties.
\end{itemize}

Enkele puntjes om in het achterhoofd te houden hoe het best  \acrshort{ipa} geïmplementeerd wordt:
\begin{itemize}
	\item Beslis snel welke rol het \acrshort{ipa} model speelt in het operationele model: snap het proces en de weg die afgelegd moet worden.
	\item Ontwerp oplossingen rond het volledige \acrshort{ipa} portfolio om zo de impact te maximaliseren: \acrshort{ipa} is het sterkst wanneer alle technologieën gecombineerd worden, bedrijven moeten hier dan ook gebruik van maken en zicht niet bezig houden met slechts enkele van de technologieën.
	\item Bouw het \acrfull{mvp} snel: in plaats van jaren te werken aan het product om het perfect te krijgen en zo eventueel over tijd en/of budget te gaan, kan beter de focus gelegd worden op het afwerken van een \acrshort{mvp} die simpele taken al kan oplossen waarop \acrshort{ipa} kan toegepast worden. Dit zal namelijk al waarde creëren terwijl de meer uitgebreide versie afgewerkt wordt.
	\item bouw momentum op en leg waarde vast: denk zowel aan de kleine overwinning op de weg als aan de  lange termijn doelstellingen.
\end{itemize} \autocite{everythingIPA}

Hieruit kan makkelijk de conclusie getrokken worden dat de groei van \acrshort{rpa} in de komende jaren alleen maar zal toenemen. Dit is daarom niet noodzakelijk in de vorm van alleenstaande \acrshort{rpa} oplossingen maar ook bijvoorbeeld in tussenstappen zoals \acrshort{rpa} 2.0 of hyperautomation naar \acrshort{ipa}.