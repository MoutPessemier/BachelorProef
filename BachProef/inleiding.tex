%%=============================================================================
%% Inleiding
%%=============================================================================

\chapter{\IfLanguageName{dutch}{Inleiding}{Introduction}}
\label{ch:inleiding}

%De inleiding moet de lezer net genoeg informatie verschaffen om het onderwerp te begrijpen en in te zien waarom de onderzoeksvraag de moeite waard is om te onderzoeken. In de inleiding ga je literatuurverwijzingen beperken, zodat de tekst vlot leesbaar blijft. Je kan de inleiding verder onderverdelen in secties als dit de tekst verduidelijkt. Zaken die aan bod kunnen komen in de inleiding~\autocite{Pollefliet2011}:

%\begin{itemize}
%  \item context, achtergrond
%  \item afbakenen van het onderwerp
%  \item verantwoording van het onderwerp, methodologie
%  \item probleemstelling
%  \item onderzoeksdoelstelling
%  \item onderzoeksvraag
%  \item \ldots
%\end{itemize}

Sinds de eerste industriële revolutie van 1750 zijn mensen op zoek naar hoe we het best verschillende processen kunnen optimaliseren en automatiseren. Er zijn al een aantal oplossingen geweest, zoals de stoommachine in de eerste industriële revolutie, elektronische apparaten tijdens de tweede industriële revolutie en natuurlijk een inleiding tot computers door Alan Turing's machine bij de derde industriële revolutie. De volgende stap in deze (r)evolutie, de vierde en huidige industriële revolutie, is dan ook Robotic Proces Automation, ook wel RPA genoemd. \autocite{indusRev}

RPA is waarschijnlijk het snelst en meest efficiënte pad naar de digitale transformatie. Om beter te begrijpen waarom RPA hier juist zo goed in is, zal eerst besproken worden wat het is en kan doen. RPA is een technologie die computer software toestaat acties die normaal gezien uitgevoerd worden door mensen te gaan simuleren of nabootsen en deze acties te integreren met digitale systemen.

RPA robots, de software agenten die deze acties op zich nemen, kunnen onder andere data vastleggen, applicaties uitvoeren, antwoorden versturen, beslissingen maken gebaseerd op voorgedefinieerde regels en communiceren met andere systemen. RPA is bedoeld voor processen die zeer sterk gereguleerd, repetitief en weinig uitzonderingen hebben.

\section{\IfLanguageName{dutch}{Probleemstelling}{Problem Statement}}
\label{sec:probleemstelling}

%Uit je probleemstelling moet duidelijk zijn dat je onderzoek een meerwaarde heeft voor een concrete doelgroep. De doelgroep moet goed gedefinieerd en afgelijnd zijn. Doelgroepen als ``bedrijven,'' ``KMO's,'' systeembeheerders, enz.~zijn nog te vaag. Als je een lijstje kan maken van de personen/organisaties die een meerwaarde zullen vinden in deze bachelorproef (dit is eigenlijk je steekproefkader), dan is dat een indicatie dat de doelgroep goed gedefinieerd is. Dit kan een enkel bedrijf zijn of zelfs één persoon (je co-promotor/opdrachtgever).

Als werknemer heel de tijd dezelfde, repetitieve taak uitvoeren is saai en vervelend werk. Als men daarbij de focus dan nog eens verliest is het ook zeer makkelijk om fouten te maken tijdens het uitvoeren van dit proces. Door het gebrek aan verantwoordelijkheid in de werknemer in combinatie met het herhaaldelijk uitvoeren van dit ene proces, verhoogt de kans op burn-outs. Om dit alles tegen te gaan kan beroep gedaan worden op Robotic Process Automation. Hierbij zal RPA voorgedefinieerde taken volledig en foutloos gaan automatiseren wat er voor zorgt dat werknemers hun tijd niet meer moeten opofferen om deze lastige taken uit te voeren maar in plaats daarvan aan nuttige en belangrijke taken kunnen werken. Dit heeft niet alleen inpakt op het mentaal welzijn van de werknemer maar ook op de kost die verminderd wordt door zo'n processen te automatiseren. Dit kan voor een bedrijf met zo een soort processen zeer voordelig uitkomen.

\section{\IfLanguageName{dutch}{Onderzoeksvraag}{Research question}}
\label{sec:onderzoeksvraag}

%Wees zo concreet mogelijk bij het formuleren van je onderzoeksvraag. Een onderzoeksvraag is trouwens iets waar nog niemand op dit moment een antwoord heeft (voor zover je kan nagaan). Het opzoeken van bestaande informatie (bv. ``welke tools bestaan er voor deze toepassing?'') is dus geen onderzoeksvraag. Je kan de onderzoeksvraag verder specifiëren in deelvragen. Bv.~als je onderzoek gaat over performantiemetingen, dan 
Als voornaamste punt zal onderzocht worden hoe efficiënt RPA werkelijk is. Daarnaast zal ook gekeken worden welke kennis en vaardigheden nodig zijn om een RPA implementatie op te zetten. Een derde onderwerp gaat over de betrouwbaarheid van zo een systeem. Hoe fout tolerant is RPA werkelijk? Hoe zit het met de up-time (hoe lang ze werken en rap ze herstelt worden na een fout) van de software agenten? Wat gebeurd er als de applicatie vast loopt?

\section{\IfLanguageName{dutch}{Onderzoeksdoelstelling}{Research objective}}
\label{sec:onderzoeksdoelstelling}

%Wat is het beoogde resultaat van je bachelorproef? Wat zijn de criteria voor succes? Beschrijf die zo concreet mogelijk. Gaat het bv. om een proof-of-concept, een prototype, een verslag met aanbevelingen, een vergelijkende studie, enz.
Uit dit onderzoek zullen de voor en nadelen van RPA aan het licht komen. Welke taken geautomatiseerd en/of verbeterd kunnen worden en welke niet.
Een schatting zal gemaakt worden rond de hoeveelheid geld dat een bedrijf kan uitsparen door een RPA implementatie. Hierbij moet dan natuurlijk rekening gehouden worden met de complexiteit en leercurve om zo een systeem op te zetten. Ook dit zal aan het licht gebracht worden.

\section{\IfLanguageName{dutch}{Opzet van deze bachelorproef}{Structure of this bachelor thesis}}
\label{sec:opzet-bachelorproef}

% Het is gebruikelijk aan het einde van de inleiding een overzicht te
% geven van de opbouw van de rest van de tekst. Deze sectie bevat al een aanzet
% die je kan aanvullen/aanpassen in functie van je eigen tekst.

De rest van deze bachelorproef is als volgt opgebouwd:

In Hoofdstuk~\ref{ch:stand-van-zaken} wordt een overzicht gegeven van de stand van zaken binnen het onderzoeksdomein, op basis van een literatuurstudie.

In Hoofdstuk~\ref{ch:methodologie} wordt de methodologie toegelicht en worden de gebruikte onderzoekstechnieken besproken om een antwoord te kunnen formuleren op de onderzoeksvragen.

% TODO: Vul hier aan voor je eigen hoofstukken, één of twee zinnen per hoofdstuk

In Hoofdstuk~\ref{ch:conclusie}, tenslotte, wordt de conclusie gegeven en een antwoord geformuleerd op de onderzoeksvragen. Daarbij wordt ook een aanzet gegeven voor toekomstig onderzoek binnen dit domein.