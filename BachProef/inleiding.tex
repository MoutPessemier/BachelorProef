%%=============================================================================
%% Inleiding
%%=============================================================================

\chapter{\IfLanguageName{dutch}{Inleiding}{Introduction}}
\label{ch:inleiding}
Sinds de eerste industriële revolutie (1750) zijn mensen op zoek naar hoe we het best verschillende processen kunnen optimaliseren en automatiseren. Er zijn al een aantal oplossingen geweest, zoals de stoommachine in de eerste industriële revolutie, elektronische apparaten tijdens de tweede industriële revolutie en natuurlijk een inleiding tot computers door Alan Turing's machine bij de derde industriële revolutie. De volgende stap in deze (r)evolutie is de vierde en huidige industriële revolutie waarvan \acrlong{rpa}, ook wel \acrshort{rpa} genoemd, deel is. \autocite{indusRev}

\acrshort{rpa} is waarschijnlijk het snelst en meest efficiënte pad naar de digitale transformatie. Om beter te begrijpen waarom \acrshort{rpa} hier juist zo goed in is, zal eerst besproken worden wat het is en kan doen. \acrshort{rpa} is een technologie die computer software toestaat acties die normaal gezien uitgevoerd worden door mensen te gaan simuleren of nabootsen en deze acties te integreren met digitale systemen.

\acrshort{rpa} robots, de software agenten die deze acties op zich nemen, kunnen onder andere data vastleggen, applicaties uitvoeren, antwoorden versturen, beslissingen maken gebaseerd op voorgedefinieerde regels en communiceren met andere systemen. \acrshort{rpa} is bedoeld voor processen die zeer sterk gereguleerd, repetitief en weinig uitzonderingen hebben.

\section{\IfLanguageName{dutch}{Probleemstelling}{Problem Statement}}
\label{sec:probleemstelling}
Als werknemers continue repetitieve administratieve taken moeten uitvoeren kan dit leiden in een onuitdagende werkervaring. Als men daarbij de focus dan nog eens verliest is het ook zeer makkelijk om fouten te maken tijdens het uitvoeren van dit proces. Door het gebrek aan verantwoordelijkheid in de werknemer in combinatie met het herhaaldelijk uitvoeren van dit ene proces, verhoogt de kans op burn-outs. Om dit alles tegen te gaan kan beroep gedaan worden op \acrlong{rpa}. Hierbij zal \acrshort{rpa} voorgedefinieerde taken volledig en foutloos gaan automatiseren wat er voor zorgt dat werknemers hun tijd niet meer moeten opofferen om deze lastige taken uit te voeren maar in plaats daarvan aan nuttige en belangrijke taken kunnen werken. Dit heeft niet alleen inpakt op het mentaal welzijn van de werknemer maar ook op de kost die verminderd wordt door zo'n processen te automatiseren. Dit kan voor een bedrijf met zo een soort processen zeer voordelig uitkomen. Maar in de zee van providers kan het moeilijk worden om te weten welke nu de geschikte \acrshort{rpa} provider is.

\section{\IfLanguageName{dutch}{Onderzoeksvraag}{Research question}}
\label{sec:onderzoeksvraag}
Hoe makkelijk of juist hoe moeilijk is het om een \acrshort{rpa} workflow te integreren met een eigen webapplicatie? Welke \acrshort{rpa} provider maakt deze taak het makkelijkst en hoe zit het dan met de tijd en kost nodig voor zo een workflow uit te werken?


\section{\IfLanguageName{dutch}{Onderzoeksdoelstelling}{Research objective}}
\label{sec:onderzoeksdoelstelling}
Uit het onderzoek zal een duidelijk beeld naar boven komen wat de voor en nadelen zijn van enkele gekozen \acrshort{rpa} providers om een integratie met een webapplicatie te voorzien. Hierbij zal rekening gehouden worden met kost van de service, implementatie tijd en gemak van integreren.

\section{\IfLanguageName{dutch}{Opzet van deze bachelorproef}{Structure of this bachelor thesis}}
\label{sec:opzet-bachelorproef}
De rest van deze bachelorproef is als volgt opgebouwd:

In Hoofdstuk~\ref{ch:stand-van-zaken} wordt een overzicht gegeven van de stand van zaken binnen het onderzoeksdomein, op basis van een literatuurstudie.

In Hoofdstuk~\ref{ch:methodologie} wordt de methodologie toegelicht en worden de gebruikte onderzoekstechnieken besproken om een antwoord te kunnen formuleren op de onderzoeksvragen.

In Hoofdstuk~\ref{ch:conclusie}, tenslotte, wordt de conclusie gegeven en een antwoord geformuleerd op de onderzoeksvragen. Daarbij wordt ook een aanzet gegeven voor toekomstig onderzoek binnen dit domein.