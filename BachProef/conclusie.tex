%%=============================================================================
%% Conclusie
%%=============================================================================
\chapter{Conclusie}
\label{ch:conclusie}
De verschillende platformen omvatten ongeveer dezelfde functionaliteiten, bij de ene al wat beter uitgewerkt als bij de ander. Zo heeft elke provider een workspace om de verschillende \gls{workflow} s in te implementeren en een platform waarop bots kunnen beheerd en toegewezen worden aan jobs om op voorgedefinieerde momenten uitgevoerd te worden. Ook liggen de \acrshort{ai} capaciteiten van de platformen binnen een \gls{workflow} dicht bijeen. Hierbij moet wel gezegd worden dat platformen zoals UiPath of Automation Anywhere nog eens een aparte service aanbieden die zich focust op het integreren van \acrshort{ai} binnen een \gls{workflow}.

Al deze gelijkenissen bewijst nog maar eens de moeilijkheid om de juiste \acrshort{rpa} provider te kiezen tussen de zee providers.

Uit het prijzenonderzoek komt het feit naar boven dat \acrshort{rpa} duur is. De meeste start-ups of kleine bedrijven beschikken niet over de financiële capaciteit om een \acrshort{rpa} oplossing te implementeren en te ondersteunen. Ook komt naar boven dat de race tussen de grote drie spelers (UiPath, Automation Anywhere en Blue Prism) als maar belangrijker wordt en dat de kleine spelers van de markt worden gestoten. Dit in combinatie met de scores die de verschillende platformen behaald hebben levert een eerste beeld op van waar welke provider zich bevindt en wat de voor- en nadelen zijn van de onderzochte providers.

Een eerste conclusie die getrokken kan worden is dat UiPath ver boven de rest uitsteekt. Zowel in score als uit het marktonderzoek is af te leiden dat UiPath de meest gebruikte en meest actieve \acrshort{rpa} provider is en dat Automation Anywhere een sterke tweede opvolger is. Dit kan alleen maar bevestigd worden door te werken met de platformen en een oplossing te voorzien.

Een tweede conclusie is dat niet alle kleine providers het slecht doen. Intellibot, ondanks de beperkte community en de niet zo overzichtelijke worflows, voelt nog steeds intuïtief en vlot aan en komt in de buurt van de grotere providers.

Als er dan gekeken wordt naar hoe makkelijk of moeilijk het is om een eigen \acrshort{api} of applicatie te integreren met de provider, dan ziet men dat dit in de meeste gevallen makkelijk te integreren valt mits het schrijven van code. De werknemers die dus geen code kunnen schrijven, zullen hier wel problemen mee ervaren.

Algemeen kan gezegd worden dat UiPath aangeraden wordt als beste \acrshort{rpa} provider en dat Automation Anywhere op de tweede plaats komt. Ook Intellibot scoort hier niet slecht, al voelt het aan als een mindere versie van UiPath.

Langs de andere kant was de teleurstelling groot bij het gebruiken van WorkFusion en Mircosoft Flow. WorkFusion heeft nog veel werk als ze ooit willen concurreren met de grote spelers. Het hele platform, de community en de communicatie achter WorkFusion waren teleurstellend en/of zaten niet goed ineen. Eigen ervaring bevestigd dit.\\
Bij Microsoft Flow zat de teleurstelling dan weer in de werkwijze en beperkte mogelijkheden van het platform. Zo kan bijna niets bereikt worden zonder een premium versie en het toevoegen van eigen activiteiten is ook een heel proces. Voor iemand die niet familiair is met de hele Microsoft suite is dit een heuse taak om te begrijpen hoe de samenwerking van alle verschillende platformen, nodig om één eigen geschreven \gls{activiteit} toe te voegen, ineen zit. Laat staan het succesvol implementeren en verbinden met elkaar.

Als laatste punt werd gekeken naar de tijd per provider, nodig om een eerste proces succesvol te implementeren. Voor UiPath en Automation Anywhere is dit gelukt op één tot twee dagen (van ongeveer 7:30 uur). Voor Intellibot zijn drie tot vijf dagen gebruikt geweest. Bij WorkFusion was er na vijf dagen nog steeds geen werkende oplossing. Voor Flow om de hele suite te leren gebruiken is ongeveer een twee tal weken nodig geweest.