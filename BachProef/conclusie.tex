%%=============================================================================
%% Conclusie
%%=============================================================================

\chapter{Conclusie}
\label{ch:conclusie}

% TODO: Trek een duidelijke conclusie, in de vorm van een antwoord op de
% onderzoeksvra(a)g(en). Wat was jouw bijdrage aan het onderzoeksdomein en
% hoe biedt dit meerwaarde aan het vakgebied/doelgroep? 
% Reflecteer kritisch over het resultaat.
% Had je deze uitkomst verwacht? Zijn er zaken die nog
% niet duidelijk zijn?
% Heeft het onderzoek geleid tot nieuwe vragen die uitnodigen tot verder 
%onderzoek?
Over het algemeen zijn er, ondanks de kleine verschillen tussen de verschillende platformen, ook veel gelijkenissen. Zo heeft elk platform wel een bepaalde vorm om bots te managen. Dit zit misschien onder een andere naam of op een andere plaats als waar de workflows gemaakt worden, maar over het algemeen heeft dit wel dezelfde features. Ook zijn de \acrshort{ai} capaciteiten over het algemeen vrij gelijk verdeeld tussen de verschillende providers. Dit bewijst nog maar eens dat de keuze naar een \acrshort{rpa} provider geen gemakkelijke is in de zee van aanbieders.\\
Uit het prijzenonderzoek blijkt dat \acrshort{rpa} over het algemeen duur is. Zelfs de kleinere platformen vragen een pak geld. Op lange termijn zal dit ook wel voordelig uitdraaien maar hebben bedrijven hier het geld wel voor? [...]

Uit de scores die de verschillende platformen behaald hebben kan een eerste conclusie getrokken worden. Hierbij steekt UiPath ver boven de rest uit. Over het algemeen zit UiPath het sterkst in de markt met hun platform. Dit wordt alleen maar versterkt door er mee te werken. Het platform voelde zeer intuïtief aan en er waren over het algemeen het minst problemen mee. Als bedrijven dan toch de stap willen zetten en er het geld voor hebben, wordt UiPath sterk aangeraden.\\
Als een soortgelijke ervaring gewenst is maar die toch goedkoper uitkomt, wordt IntelliBot aangeraden. Buiten de IntelliBot Studio voelde deze veruit het best aan tussen de kleine providers. Er wordt op dit moment nog gewerkt aan het platform maar heeft ook al veel te bieden. De manier van werken is even wennen maar eens onder de knie kan een workflow snel opgebouwd en gepubliceerd worden. \\
Automation Anywhere is zeker geen slechte optie. Ook zij bieden een goed product aan. Het kwam neer op eigen ervaring waarbij de voorkeur uit ging naar UiPath boven Automation Anywhere.

Voor de slechter scorende gevallen is het gevoel minder positief. Bij Microsoft Flow was de teleurstelling groot. Ongeveer 90\% van het platform zit vast achter een premium licentie. Daarbij is de support van Microsoft gekend voor hun slechte kwaliteit en de prijs speelt zeker ook niet in het voordeel. Zelfs al zit een bedrijf reeds in de Microsoft suite (met Azure en andere services) wordt het als nog afgeraden om Power Automate te gebruiken voor het implementeren van \acrshort{rpa} solutions.\\
Voor WorkFusion is het dan weer een ander verhaal. WorkFusion werkt goed voor zeer basis workflows die weinig tot geen logica nodig hebben. Van zodra dit criteria overschreven wordt, is het rap zeer moeilijk om een geschikte implementatie te voorzien. Het gebruik van basis Groovy zonder mogelijke uitbreidingen en het ontbreken van syntax highlighting en autocomplete beperken de mogelijkheid het schrijven van eigen activiteiten sterk.

[...]