%%=============================================================================
%% Conclusie
%%=============================================================================
\chapter{Conclusie}
\label{ch:conclusie}
De verschillende platformen omvatten ongeveer dezelfde functionaliteiten, bij de ene al wat beter uitgewerkt als bij de ander. Zo heeft elke provider een workspace om de verschillende \gls{workflow}s te implementeren en een platform waarop bots kunnen beheerd en toegewezen worden aan jobs om op voorgedefinieerde momenten uitgevoerd te worden. Ook liggen de \acrshort{ai} capaciteiten van de platformen binnen een \gls{workflow} dicht bijeen. Hierbij moet wel gezegd worden dat platformen zoals UiPath of Automation Anywhere nog eens een aparte service aanbieden die zich focust op het integreren van \acrshort{ai} binnen een \gls{workflow}.

Al deze gelijkenissen bewijzen nog maar eens de moeilijkheid om de juiste \acrshort{rpa} provider te kiezen uit de zee providers.

Uit het prijzenonderzoek kunnen we besluiten dat \acrshort{rpa} duur is. De meeste start-ups of kleine bedrijven beschikken niet over de financiële capaciteit om een \acrshort{rpa} oplossing te implementeren en ondersteunen. De race tussen de grote drie spelers (UiPath, Automation Anywhere en Blue Prism) blijkt alsmaar belangrijker te worden waardoor de kleine spelers van de markt worden gestoten. Dit in combinatie met de scores die de verschillende platformen behaald hebben levert een eerste beeld op waar elke provider zich bevindt en wat hun voor- en nadelen zijn.

Een eerste conclusie die getrokken kan worden is dat UiPath ver boven de rest uitsteekt. Zowel in score als uit het marktonderzoek is af te leiden dat UiPath de meest gebruikte en meest actieve \acrshort{rpa} provider is en dat Automation Anywhere een sterke tweede plaats inneemt. Door te werken met deze platformen wordt deze stelling bevestigd.

Een tweede conclusie is dat niet alle kleine providers het slecht doen. Intellibot, ondanks de beperkte community en de niet zo overzichtelijke worflows, voelt nog steeds intuïtief en vlot aan en komt in de buurt van de grotere providers.

Als er gekeken wordt naar hoe makkelijk of moeilijk het is om een eigen \acrshort{api} of applicatie te integreren met de provider, ziet men dat dit in de meeste gevallen makkelijk te integreren valt mits het schrijven van code. De werknemers die dus geen code kunnen schrijven, zullen problemen ervaren.

Algemeen kan gezegd worden dat UiPath aangeraden wordt als beste \acrshort{rpa} provider en dat Automation Anywhere op de tweede plaats komt. Ook Intellibot scoort hier niet slecht, al voelt het aan als een mindere versie van UiPath.

Langs de andere kant was de teleurstelling groot bij het gebruiken van WorkFusion en Mircosoft Flow. WorkFusion heeft nog veel werk als ze ooit willen concurreren met de grote spelers. Het hele platform, de community en de communicatie achter WorkFusion waren teleurstellend en waren niet goed opgebouwd. Eigen ervaring bevestigt dit.\\
Microsoft Flow scoorde minder op het valk van werkwijze en mogelijkheden van het platform. Zonder een premium versie kan bijna niets bereikt worden. Het toevoegen van eigen activiteiten is zeer omslachtig. Voor iemand die niet gewend is om te werken met de Microsoft suite is het een heuse taak om te begrijpen hoe de samenwerking van alle verschillende platformen in mekaar zit. Laat staan het succesvol implementeren van een zelfgeschreven activiteit en deze gebruiken binnen Microsoft Flow.

Als laatste punt werd gekeken naar de tijd per provider, nodig om een eerste proces succesvol te implementeren. Voor UiPath en Automation Anywhere is dit gelukt op één tot twee dagen (van ongeveer 7:30 uur). Voor Intellibot zijn drie tot vijf dagen gebruikt geweest. Bij WorkFusion was er na vijf dagen nog steeds geen werkende oplossing. Voor Microsoft Flow waren een tweetal weken nodig om de hele suite te leren gebruiken.