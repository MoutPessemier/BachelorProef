%===============================================================================
% LaTeX sjabloon voor de bachelorproef toegepaste informatica aan HOGENT
% Meer info op https://github.com/HoGentTIN/bachproef-latex-sjabloon
%===============================================================================

\documentclass{bachproef-tin}

\usepackage{hogent-thesis-titlepage}
\usepackage{graphicx}
\graphicspath{{../Assets/}}
\usepackage{url}
\usepackage{makecell}
\usepackage{listings}
\usepackage{rotating}
\usepackage{glossaries}

\makeglossaries

\newglossaryentry{no_code_rpa}{name={No-Code RPA},description={Een manier voor het implementeren van processen, gebruik makend van RPA maar zonder iets te coderen}}

\newglossaryentry{low_code_rpa}{name={Low-Code RPA},description={Een manier voor het implementeren van processen, gebruik makend van RPA waarbij een minimum aan code geschreven wordt. Dit gaat meestal over een aantal if-statements of een loop conditie}}

\newglossaryentry{activiteit}{name={activiteit},description={Een activiteit is een taak die door een bot kan worden uitgevoerd.}}

\newglossaryentry{workflow}{name={workflow},description={Een workflow is een opeenvolging van activiteiten die uitgevoerd worden door de rpa bot. Één workflow stelt meestal één proces voor.}}

\newacronym{rpa}{RPA}{Robotic Process Automation}
\newacronym{ai}{AI}{Artificiële Inteligentie}
\newacronym{ml}{ML}{Machine Learning}
\newacronym{ipa}{IPA}{Intelligent Process Automation}
\newacronym{bpa}{BPA}{Business Process Automation}
\newacronym{hr}{HR}{Human Resources}
\newacronym{bpm}{BPM}{Business Process Management}
\newacronym{ocr}{OCR}{Optical Character Recognition}
\newacronym{nlp}{NLP}{Natural Language Processing}
\newacronym{nlg}{NLG}{Natural Language Generation}
\newacronym{cmp}{CMP}{Cognitive Modeling Platform}
\newacronym{idr}{IDR}{Intelligent Digital Robots}
\newacronym{mvp}{MVP}{Minimum Viable Product}
\newacronym{api}{API}{Application Programming Interface}
\newacronym{adp}{ADP}{Automated Document Processing}
\newacronym{ide}{IDE}{Integrated Development Environment}
\newacronym{http}{HTTP}{Hyper Text Transfer Protocol}
\newacronym{ui}{UI}{User Interface}
\newacronym{json}{JSON}{JavaScript Object Notation}
%%---------- Documenteigenschappen ---------------------------------------------

\title{Robotic Process Automation, automatisering van morgen, vandaag.}

\author{Mout Pessemier}

\promotor{Johan Decorte}

\copromotor{Laurens Lavaert}

\instelling{Faktion}

\academiejaar{2019-2020}

\examenperiode{2}

%===============================================================================
% Inhoud document
%===============================================================================

\begin{document}

%---------- Titelblad ----------------------------------------------------------
\inserttitlepage

%---------- Samenvatting, voorwoord --------------------------------------------
\usechapterimagefalse
%%=============================================================================
%% Voorwoord
%%=============================================================================

\chapter*{\IfLanguageName{dutch}{Woord vooraf}{Preface}}
\label{ch:voorwoord}

Voor ik aan deze bachelorproef begon had ik absoluut geen idee wat \acrlong{rpa} was en wat er mee bereikt kon worden. Ik zag het dan ook als een uitdaging die ik niet links kon laten liggen om mij volledig in dit veld te gooien en de wereld van automatisatie beter te begrijpen. Wat is nu het voordeel van \acrshort{rpa} boven het schrijven van automatisatie scripts, wie bevindt zich op deze markt, hoe groot is deze markt en zit er een toekomst in?

Voordat verder gegaan wordt met deze thesis, wil ik eerst enkele mensen bedanken. Deze bachelorproef zou nooit tot stand gekomen zijn zonder de hulp van enkele mensen.\\
Eerst en vooral zou ik mijn co-promotor, Laurens Laveart, willen bedanken voor de gedetailleerde
opvolging van mijn bachelorproef. De feedback over de inhoud en de vele vragen die ik heb kunnen stellen waren cruciaal voor het succes van deze proef.\\
Verder wil ik mijn promotor, Johan Decorte, heel erg bedanken voor de uren die
gespendeerd zijn aan de opvolging van en het in goede banen sturen van deze paper.\\
Ik kan natuurlijk Jo Cijnsmans en Niels Van Weereld, het sales en marketing team van Faktion niet vergeten. Zij hebben mij ondersteund in het opstellen van het marktonderzoek dat terug te vinden is als bijlage.\\
Ik zou graag ook enkele mensen bedanken die mijn bachelorproef nog eens nagelezen hebben zoals Jonathan Callewaert, [...] en mijn ouders.\\
Daarnaast zou ik Joeri Van Steen en Filip Martens van Faktion willen bedanken voor de mogelijkheid om op hun platform en van hun resources gebruik te mogen maken om mijn thesis tot een goed eind te brengen. \\
Bovendien wil ik nog iedereen bedanken die op de één of andere manier geholpen heeft met mijn bachelorproef.

Als laatste wens ik u nog een boeiende lectuur.


%%=============================================================================
%% Samenvatting
%%=============================================================================
\chapter*{\IfLanguageName{dutch}{Samenvatting}{Abstract}}
Wie zoekt naar \acrshort{rpa}-aanbieders op het internet, krijgt een lijst van providers. Iedere provider heeft voor- en nadelen. Om hieruit het juiste platform uit te kiezen dat past bij de bedrijfscultuur en mentaliteit is geen gemakkelijke taak. Er zijn vele aspecten die overwogen moeten worden om deze keuze succesvol te maken. Daarom is in deze thesis onderzoek uitgevoerd geweest naar enkele van deze providers om hun sterke en zwakke punten in kaart te brengen om zo de juiste keuze te maken.\\
Het onderzoek gaat na bij vijf providers (UiPath, Automation Anywhere, WorkFusion, IntelliBot en Microsoft Flow) waar ergens ze zich situeren in de markt, waarin ze uitblinken als platform en waaraan nog verbeteringen nodig zijn. Bij de selectie van de vijf aanbieders is rekening gehouden met criteria zoals de grootte van het bedrijf en het soort \acrshort{rpa}. Dit is gerealiseerd geweest door met de verschillende platformen te werken om een demo proces te automatiseren. Daarnaast wordt er ook contact gelegd met het bedrijf en de community achter de provider.\\
Elke aanbieder zijn resultaat staat in detail beschreven in hoofdstuk drie: Methodologie. Kort samengevat, eerst is door de lessen gegaan die aangeboden werden op het platform zelf. Nadien is het proces geautomatiseerd geweest behalve de zelfgeschreven \gls{activiteit}. Daarna is deze \gls{activiteit} geïmplementeerd en geïntegreerd geweest in de \gls{workflow}. Dit is nadien extensief getest geweest op fouten.\\
Uiteindelijk werden aan de vooropgestelde criteria punten toegewezen nadat elk platform getest was. Deze criteria zijn onderverdeeld in drie categorieën: technische-, bedrijfs- en financiële aspecten.\\
De belangrijkste conclusies van dit onderzoek zijn dat UiPath duidelijk als beste naar voor komt. Ook kan besloten worden dat de kleine providers in de toekomst waarschijnlijk uit de markt zullen verdwijnen. Dit is te wijten aan de voorspelling dat het een race zal worden tussen de drie grote providers (UiPath, Automation Anywhere en Blue Prism). Wie van de drie zal de markt uiteindelijk beheersen? Op dit moment ziet het er naar uit dat UiPath zal winnen.\\
Dit onderzoek is alles behalve volledig. Er zijn vele andere providers die nog niet onderzocht geweest zijn en mogelijks ook het vermelden waard zijn. Daarnaast zijn er ook nog andere criteria die onderzocht kunnen worden. Als laatste kan ook een diepgaander onderzoek uitgevoerd worden naar de race tussen de 'Big Three'.

%---------- Inhoudstafel -------------------------------------------------------
\pagestyle{empty}
\tableofcontents
\cleardoublepage
\pagestyle{fancy}

%---------- Lijst figuren, afkortingen, ... ------------------------------------
\listoffigures
\listoftables
\printglossaries

%---------- Kern ---------------------------------------------------------------
%%=============================================================================
%% Inleiding
%%=============================================================================

\chapter{\IfLanguageName{dutch}{Inleiding}{Introduction}}
\label{ch:inleiding}
Sinds de eerste industriële revolutie van 1750 zijn mensen op zoek naar hoe we het best verschillende processen kunnen optimaliseren en automatiseren. Er zijn al een aantal oplossingen geweest, zoals de stoommachine in de eerste industriële revolutie, elektronische apparaten tijdens de tweede industriële revolutie en natuurlijk een inleiding tot computers door Alan Turing's machine bij de derde industriële revolutie. De volgende stap in deze (r)evolutie, de vierde en huidige industriële revolutie, is dan ook \acrlong{rpa}, ook wel \acrshort{rpa} genoemd. \autocite{indusRev}

\acrshort{rpa} is waarschijnlijk het snelst en meest efficiënte pad naar de digitale transformatie. Om beter te begrijpen waarom \acrshort{rpa} hier juist zo goed in is, zal eerst besproken worden wat het is en kan doen. \acrshort{rpa} is een technologie die computer software toestaat acties die normaal gezien uitgevoerd worden door mensen te gaan simuleren of nabootsen en deze acties te integreren met digitale systemen.

\acrshort{rpa} robots, de software agenten die deze acties op zich nemen, kunnen onder andere data vastleggen, applicaties uitvoeren, antwoorden versturen, beslissingen maken gebaseerd op voorgedefinieerde regels en communiceren met andere systemen. \acrshort{rpa} is bedoeld voor processen die zeer sterk gereguleerd, repetitief en weinig uitzonderingen hebben.

\section{\IfLanguageName{dutch}{Probleemstelling}{Problem Statement}}
\label{sec:probleemstelling}
Als werknemer heel de tijd dezelfde, repetitieve taak uitvoeren is saai en vervelend werk. Als men daarbij de focus dan nog eens verliest is het ook zeer makkelijk om fouten te maken tijdens het uitvoeren van dit proces. Door het gebrek aan verantwoordelijkheid in de werknemer in combinatie met het herhaaldelijk uitvoeren van dit ene proces, verhoogt de kans op burn-outs. Om dit alles tegen te gaan kan beroep gedaan worden op \acrlong{rpa}. Hierbij zal \acrshort{rpa} voorgedefinieerde taken volledig en foutloos gaan automatiseren wat er voor zorgt dat werknemers hun tijd niet meer moeten opofferen om deze lastige taken uit te voeren maar in plaats daarvan aan nuttige en belangrijke taken kunnen werken. Dit heeft niet alleen inpakt op het mentaal welzijn van de werknemer maar ook op de kost die verminderd wordt door zo'n processen te automatiseren. Dit kan voor een bedrijf met zo een soort processen zeer voordelig uitkomen. Maar in de zee van providers kan het moeilijk worden om te weten welke nu de geschikte \acrshort{rpa} provider is.

\section{\IfLanguageName{dutch}{Onderzoeksvraag}{Research question}}
\label{sec:onderzoeksvraag}
Hoe makkelijk of juist hoe moeilijk is het om een \acrshort{rpa} workflow te integreren met een eigen webapplicatie? Welke \acrshort{rpa} provider maakt deze taak het makkelijkst en hoe zit het dan met de tijd en kost nodig voor zo een workflow uit te werken?


\section{\IfLanguageName{dutch}{Onderzoeksdoelstelling}{Research objective}}
\label{sec:onderzoeksdoelstelling}
Uit het onderzoek zal een duidelijk beeld naar boven komen wat de voor en nadelen zijn van enkele gekozen \acrshort{rpa} providers om een integratie met een webapplicatie te voorzien. Hierbij zal rekening gehouden worden met kost van de service, implementatie tijd en gemak van integreren.

\section{\IfLanguageName{dutch}{Opzet van deze bachelorproef}{Structure of this bachelor thesis}}
\label{sec:opzet-bachelorproef}
De rest van deze bachelorproef is als volgt opgebouwd:

In Hoofdstuk~\ref{ch:stand-van-zaken} wordt een overzicht gegeven van de stand van zaken binnen het onderzoeksdomein, op basis van een literatuurstudie.

In Hoofdstuk~\ref{ch:methodologie} wordt de methodologie toegelicht en worden de gebruikte onderzoekstechnieken besproken om een antwoord te kunnen formuleren op de onderzoeksvragen.

In Hoofdstuk~\ref{ch:conclusie}, tenslotte, wordt de conclusie gegeven en een antwoord geformuleerd op de onderzoeksvragen. Daarbij wordt ook een aanzet gegeven voor toekomstig onderzoek binnen dit domein.
\chapter{\IfLanguageName{dutch}{Stand van zaken}{State of the art}}
\label{ch:stand-van-zaken}

% Tip: Begin elk hoofdstuk met een paragraaf inleiding die beschrijft hoe
% dit hoofdstuk past binnen het geheel van de bachelorproef. Geef in het
% bijzonder aan wat de link is met het vorige en volgende hoofdstuk.

% Pas na deze inleidende paragraaf komt de eerste sectiehoofding.

%Dit hoofdstuk bevat je literatuurstudie. De inhoud gaat verder op de inleiding, maar zal het onderwerp van de bachelorproef *diepgaand* uitspitten. De bedoeling is dat de lezer na lezing van dit hoofdstuk helemaal op de hoogte is van de huidige stand van zaken (state-of-the-art) in het onderzoeksdomein. Iemand die niet vertrouwd is met het onderwerp, weet nu voldoende om de rest van het verhaal te kunnen volgen, zonder dat die er nog andere informatie moet over opzoeken \autocite{Pollefliet2011}.

%Je verwijst bij elke bewering die je doet, vakterm die je introduceert, enz. naar je bronnen. In \LaTeX{} kan dat met het commando \texttt{$\backslash${textcite\{\}}} of \texttt{$\backslash${autocite\{\}}}. Als argument van het commando geef je de ``sleutel'' van een ``record'' in een bibliografische databank in het Bib\LaTeX{}-formaat (een tekstbestand). Als je expliciet naar de auteur verwijst in de zin, gebruik je \texttt{$\backslash${}textcite\{\}}.
%Soms wil je de auteur niet expliciet vernoemen, dan gebruik je \texttt{$\backslash${}autocite\{\}}. In de volgende paragraaf een voorbeeld van elk.

%\textcite{Knuth1998} schreef een van de standaardwerken over sorteer- en zoekalgoritmen. Experten zijn het erover eens dat cloud computing een interessante opportuniteit vormen, zowel voor gebruikers als voor dienstverleners op vlak van informatietechnologie~\autocite{Creeger2009}.

%\lipsum[7-20]

RPA is een onderdeel van business process automation (BPA), een overkoepelende term gebruikt voor het beschrijven van technologieën die activiteiten en workflows waaruit business taken bestaan uitvoeren met zo min mogelijk menselijke tussenkomst. \autocite{everythingRPA}

RPA kan gecombineerd worden samen met artificiële intelligentie om zo langere en moeilijkere taken op zich te nemen. Hierdoor worden ze door sommigen beschreven als zelf-verbeterende digitale werkkrachten. Enkele velden waarbinnen RPA kan gecombineerd worden met AI zijn optical character recognition en natural language generation (NLG). \autocite{everythingRPA}

\section{Algemene informatie}

\subsection{Waarvoor kan RPA gebruikt worden}
Enkele taken waarvoor RPA kan gebruikt worden:
\begin{itemize}
	\item data overzetten van de ene applicatie naar de andere
	\item processen automatiseren
	\item website scraping
	\item data verzamelen en analyseren
	\item reporting
	\item emails verzenden
	\item werken met spreadsheets, PDFs, texteditors, ...
	\item ...
\end{itemize}
\autocite{everythingRPA} \autocite{idrRPA}

\subsection{In welke sectoren wordt RPA gebruikt}
Onderdelen van een onderneming waarbinnen deze taken kunnen gebruikt worden:
\begin{itemize}
	\item customer service: validatie van handtekeningen, uploaden van ingescande documenten, valideren van informatie voor automatische aanvaarding of afwijzing
	\item accounting: procedure to pay, accounting, belastingen, ...
	\item human resources (HR): payroll, time management, recruitment, ...
	\item IT management en services: source-code control management, incident management, optimailseren van email notificaties, ...
	\item supply chain management: order processing, payment processing, monitoring van de stock, ...
\end{itemize}
\autocite{everythingRPA}

\subsection{Welke processen kunnen geautomatiseerd worden}
RPA is kan niet gebruikt worden om zo maar eender welk business proces te gaan automatiseren. Een proces toont best een aantal karakteristieken die het geschikt maken om te automatiseren:
\begin{itemize}
	\item repetitief
	\item gebaseerd op gestructureerde, digitale data
	\item duidelijk afgebakende regels met weinig tot geen uitzonderingen
	\item error-prone als het uitgevoerd wordt door een werknemer
	\item tijdsgebonden
\end{itemize}

\section{Verschillen met soortgelijke technologieën}

\subsection{Gewone automatisatie}
Het grote verschil tussen RPA en gewone automatisatie is de kennis van coderen die nodig is om het doel te bereiken. Bij gewone automatisatie schrijft een programmeur code die een bepaalde taak automatiseerd. Bij RPA daarentegen wordt over het algemeen weinig tot geen code gebruikt om een proces te gaan automatiseren.
\subsection{Business Process Management}
Ook is er een verschil tussen RPA en business process management (BPM). Beide willen het aantal fouten verminderen en efficiëntie verhogen maar ze doen dat op faliekant andere manieren. BPM gaat helderheid scheppen bij business processen door deze te gaan modelleren en in kaart te brengen waar RPA deze processen juist wil gaan automatiseren, daarom niet in kaart brengen.
\subsection{Intelligent Process Automation}
Als laatste is er intelligent process automation (IPA) tegenover RPA. Er is hier minder sprake van een verschil aangezien IPA RPA combineerd met BPM, AI en machine learning (ML) om processen optimaal te gaan verbeteren.
\autocite{everythingRPA}

\section{De verschillende RPA Architecturen}
Er zijn verschillende soorten RPA die kunnen gebruikt worden binnen een bedrijfscontext:

\begin{itemize}
	\item Assisted (Attended) automation: RPA loopt op een gebruiker zijn desktop om hem te helpen een proces op een snellere manier kunnen af te werken. Dit leid in het algemeen tot een vermindering van kosten en een betere klantenservice. Een groot nadeel aan deze architectuur is dat een wijziging of onregelmatigheid in de beeldscherm instellingen zoals bijvoorbeeld het wijzigen van display instellingen als lettergrootte kan er voor zorgen dat RPA faalt in het uitvoeren van de vooropgestelde taak.
	\item Unassisted (Unattended) automation: heeft geen menselijke interactie nodig, RPA draait het systeem zelf en laat pas iets weten aan de werknemer wanneer iets fout gaat en hij het zelf niet kan oplossen. Dit zijn de 24/7 digitale werkkrachten die geassocieerd worden met RPA.
	\item Hybrid RPA: Een combinatie van vorige 2 architecturen waarbij de werknemer en de RPA robot als een team samen werken, onderling taken van en naar elkaar sturen. De hybride methode overkoepelt op deze manier zowel het werk dat alleen uitgevoerd kan worden door de robot als ook het werk waarvoor menselijke interactie nodig is.
\end{itemize}

Daarnaast kan RPA op verschillende manieren geïmplementeerd worden, zo kan de low-code development mentaliteit gevolgd worden waarbij amper iets van kennis nodig is. Dit is perfect voor bedrijven met een beperkte IT kennis. Aan de andere kant kan ook veel gecodeerd worden om deze robots op te zetten.
\autocite{everythingRPA}

\section{Verschillen met soortgelijke technologieën}

\subsection{Gewone automatie}
Het grote verschil tussen RPA en gewone automatisatie is de kennis van coderen die nodig is om het doel te bereiken. Bij gewone automatisatie schrijft een programmeur code die een bepaalde taak gaat automatiseren. Bij RPA daarentegen wordt over het algemeen weinig tot geen code gebruikt om een proces te gaan automatiseren maar wordt via drag \& drop of de recorder de automatie samengesteld. \autocite{everythingRPA}

\subsection{Business Process Management}
Ook is er een verschil tussen RPA en business process management (BPM). Beide willen het aantal fouten verminderen en efficiëntie verhogen maar ze doen dat op faliekant andere manieren. BPM gaat helderheid scheppen bij business processen door deze te gaan modelleren en in kaart te brengen waar RPA deze processen juist wil gaan automatiseren, daarom niet in kaart brengen. \autocite{everythingRPA}

\subsection{Intelligent Process Automation}
Als laatste is er intelligent process automation (IPA) tegenover RPA. Er is hier minder sprake van een verschil aangezien IPA RPA combineerd met BPM, AI en machine learning (ML) om processen optimaal te gaan verbeteren. \autocite{everythingRPA}

\section{Voordelen van RPA}

\subsection{Efficiëntie}
Vanwaar komt de efficiëntie van RPA?
\begin{itemize}
	\item Robots werken 24/7: in tegenstelling tot werknemers blijven robots werken na de werkuren. Ze worden ook niet moe en moeten nooit naar de wc. Ze kunnen ook hun concentratie niet verliezen. Als al deze punten opgeteld worden, dan kan slechts 1 conclusie getrokken worden: RPA robots zijn harde werkers die nooit pauze nodig hebben.
	\item Snellere uitvoertijd: het aantal transacties die een robot kan afwerken per uur is veel meer dan het aantal dat een werknemer ooit zou kunnen afwerken binnen hetzelfde tijdskader.
	\item Het vinden van fouten in het systeem: RPA verzamelt informatie over het systeem waar het op werkt en giet dit in een analytisch overzicht. Hier kan nadien informatie uit gehaald worden waar een vertraging of opstopping in het proces zich voordoet.
	\item Meer gemotiveerde werknemers: waar RPA het proces zal verbeteren, zal ook de werknemer die verlost is van deze lastige taak meer motivatie tonen naar de leuke processen waaraan deze nu kan werken.
	\item Doe meer met minder: Waar RPA kan gezien worden als een manier om even veel werk te verrichten met minder werknemers is het beter om te kijken naar de mogelijkheden die RPA biedt om meer werk te verrichten met hetzelfde aantal werknemers wat tot een groei van de onderneming kan leiden.
\end{itemize}

\autocite{efficiencyRPA}
\subsection{Nauwkeurigheid}
Als een proces juist opgezet is, dan kunnen RPA robots dit proces blijven juist uitvoeren zonder vermoeid te raken, hierdoor daalt het foutpercentage drastisch tot dicht bij 0\%. \autocite{efficiencyRPA}

\subsection{Kostvermindering}
RPA zorgt ervoor dat de hoeveelheid werk die afgewerkt geraakt zal stijgen met eenzelfde aantal werknemers of staat toe aan een bedrijf om met minder werknemers even veel werk klaar te krijgen. \autocite{efficiencyRPA}

\subsection{Werknemer productiviteit en moraal}
Door werknemers van de saaie en repetitieve taken weg te nemen en ze te laten werken aan taken die wel degelijk belang hebben binnen een bedrijf stijgt de inzet, moraal en productiviteit die van een werknemer kan verwacht worden. \autocite{efficiencyRPA}

\subsection{Digitale transformatie}
Door het gebruik van RPA kan een bedrijf delen of hele processen gaan automatiseren zonder wijzigingen aan te brengen aan legacy-code of te investeren in integratie van back-end systemen met reeds bestaande systemen. Hierdoor kunnen ze verder door gaan in deze digitale transformatie. \autocite{efficiencyRPA}

\section{Nadelen van RPA}
Gemiddeld falen 30 tot 50\% de eerst RPA projecten. \autocite{everythingRPA} Een van de grootste redenen waarom zo veel projecten falen is het feit dat men denkt dat een RPA project een puur IT project is. Dit is niet correct, het RPA proces moet opgezet worden door werknemers die kennis hebben van RPA maar ook de werknemers die het proces snappen en de 'as-is' kunnen interpreteren om zo deze werknemers te sturen in de juiste richting.

Enkele andere veelvoorkomende risico's zijn:

\subsection{Onderschatten van human capital}
Ondanks net gezegd dat RPA niet alleen IT is, is de aanwezigheid van IT van vitaal belang om een RPA project te kunnen blijven ondersteunen/onderhouden. Iedereen kan een proces automeren maar het draaiende houden en onderhouden is al een pak technischer en dus niet voor iedereen weggelegd. \autocite{everythingRPA}

\subsection{Security risico's}
RPA brengt een aantal security risico's met zich mee. Zo kan een robot die niet goed geïmplementeerd is bijvoorbeeld frauduleuze zaken uitvoeren of toegang krijgen tot en werken met sensitieve data en beheerders privileges gaan misbruiken. Het is dus belangrijk om de processen zorgvuldig en aandachtig te gaan opstellen en zeker te controleren dat deze risico's vermeden kunnen worden. \autocite{everythingRPA} \autocite{predictionRPA}

\section{De toekomst van RPA}
We hebben natuurlijk geen glazen bol en kunnen dus niets met zekerheid zeggen over de toekomst van RPA. Wel kan gespeculeerd worden hoe deze technologie zal evolueren in de komende jaren. Experten zijn het er over het algemeen mee eens dat RPA de toekomst is van automatisering binnen IT. \autocite{futRPA}

\subsection{Enkele nummers}
RPA is ver gekomen sinds 2016. Toen lag het bedrag van gekochte RPA software rond \$73 miljoen. Dit is gestegen tot \$113 miljoen in 2017, \$153 miljoen in 2018 en \$192 miljoen in 2019. Er wordt geschat dat dit bedrag zal stijgen tot \$232 miljoen in 2020 en tegen 2021 zou dit bedrag liggen rond de \$272 miljoen. Hieruit valt duidelijk af te leiden dat er een sterke stijging in het gebruik van RPA voorlopig aan de gang is en dat dit niet direct blijkt te stoppen. Meer en meer ziet men dat RPA in de globale markt opgenomen wordt. \autocite{futRPA} Ondanks dat de enorme groei in RPA uitgaven blijft stijgen, zijn er toch enkele vragen die op komen rond de levensduur van RPA-markten. \autocite{everythingRPA}

Enkele verwachtingen:
\begin{itemize}
	\item Meer gebruik in de komende jaren
	\item combinatie met AI, cloud services en ML om RPA te verheven tot IPA
	\item combinatie met andere tools
\end{itemize}

\subsection{RPA 2.0}
RPA 2.0, ook wel intelligent digital robots (IDR) genoemd, gaat de huidige RPA gaan verbeteren door gebruik te maken van AI en ML om de robots slim te maken. Hierdoor moeten ze niet blindelings de regeltjes volgen maar kunnen ze, gebaseerd op de algoritmen en de interpretatie van de data, eigen beslissingen maken. Dit is een duidelijke stap naar het befaamde IPA. \autocite{idrRPA}

Vooral deep-learning en reinforcement learning technieken worden gebruikt om RPA 2.0 te verwezenlijken. \autocite{idrRPA}
%%=============================================================================
%% Methodologie
%%=============================================================================

\chapter{\IfLanguageName{dutch}{Methodologie}{Methodology}}
\label{ch:methodologie}

%% TODO: Hoe ben je te werk gegaan? Verdeel je onderzoek in grote fasen, en
%% licht in elke fase toe welke stappen je gevolgd hebt. Verantwoord waarom je
%% op deze manier te werk gegaan bent. Je moet kunnen aantonen dat je de best
%% mogelijke manier toegepast hebt om een antwoord te vinden op de
%% onderzoeksvraag.

Aan de hand van verchillende RPA providers zal een concreet proces geautomatiseerd worden op Metamaze, het automated document processing platform bij Faktion zelf.


\section{Voorbereiding}
Ter voorbereiding van het te automatiseren proces bij Faktion heb ik mij eerst ingewerkt op het platform waar dit proces zich voordoet. Ook heb ik de volledige 'RPA Developer Essentials' cursus aangeboden door UiPath gevolgd om een algemene basiskennis rond RPA te beheersen. Daarna is de afweging gemaakt welke RPA software providers nu zullen vergelijken worden. Hieruit zijn 5 kandidaten naar boven gekomen. Als 2 grote providers zal gekeken worden naar UiPath en Automation Anywhere. Voor de kleinere providers wordt gekeken naar AutomationEdge en WorkFusion.

Ondertussen werd het te automatiseren proces uitgewerkt. Dit proces zal worden geïmplementeerd op de verschillende  platformen.

Eens dat geschreven is zal er gekeken worden naar connectors om het RPA proces te verbinden met MetaMaze.

\section{Implementatie}
\section{Vergelijking}

%%=============================================================================
%% Conclusie
%%=============================================================================

\chapter{Conclusie}
\label{ch:conclusie}

% TODO: Trek een duidelijke conclusie, in de vorm van een antwoord op de
% onderzoeksvra(a)g(en). Wat was jouw bijdrage aan het onderzoeksdomein en
% hoe biedt dit meerwaarde aan het vakgebied/doelgroep? 
% Reflecteer kritisch over het resultaat.
% Had je deze uitkomst verwacht? Zijn er zaken die nog
% niet duidelijk zijn?
% Heeft het onderzoek geleid tot nieuwe vragen die uitnodigen tot verder 
%onderzoek?
Over het algemeen zijn er, ondanks de kleine verschillen tussen de verschillende platformen, ook veel gelijkenissen. Zo heeft elk platform wel een bepaalde vorm om bots te managen. Dit zit misschien onder een andere naam of op een andere plaats als waar de workflows gemaakt worden, maar over het algemeen heeft dit wel dezelfde features. Ook zijn de \acrshort{ai} capaciteiten over het algemeen vrij gelijk verdeeld tussen de verschillende providers. Dit bewijst nog maar eens dat de keuze naar een \acrshort{rpa} provider geen gemakkelijke is in de zee van aanbieders.\\
Uit het prijzenonderzoek blijkt dat \acrshort{rpa} over het algemeen duur is. Zelfs de kleinere platformen vragen een pak geld. Op lange termijn zal dit ook wel voordelig uitdraaien maar hebben bedrijven hier het geld wel voor? [...]

Uit de scores die de verschillende platformen behaald hebben kan een eerste conclusie getrokken worden. Hierbij steekt UiPath ver boven de rest uit. Over het algemeen zit UiPath het sterkst in de markt met hun platform. Dit wordt alleen maar versterkt door er mee te werken. Het platform voelde zeer intuïtief aan en er waren over het algemeen het minst problemen mee. Als bedrijven dan toch de stap willen zetten en er het geld voor hebben, wordt UiPath sterk aangeraden.\\
Als een soortgelijke ervaring gewenst is maar die toch goedkoper uitkomt, wordt IntelliBot aangeraden. Buiten de IntelliBot Studio voelde deze veruit het best aan tussen de kleine providers. Er wordt op dit moment nog gewerkt aan het platform maar heeft ook al veel te bieden. De manier van werken is even wennen maar eens onder de knie kan een workflow snel opgebouwd en gepubliceerd worden. \\
Automation Anywhere is zeker geen slechte optie. Ook zij bieden een goed product aan. Het kwam neer op eigen ervaring waarbij de voorkeur uit ging naar UiPath boven Automation Anywhere.

Voor de slechter scorende gevallen is het gevoel minder positief. Bij Microsoft Flow was de teleurstelling groot. Ongeveer 90\% van het platform zit vast achter een premium licentie. Daarbij is de support van Microsoft gekend voor hun slechte kwaliteit en de prijs speelt zeker ook niet in het voordeel. Zelfs al zit een bedrijf reeds in de Microsoft suite (met Azure en andere services) wordt het als nog afgeraden om Power Automate te gebruiken voor het implementeren van \acrshort{rpa} solutions.\\
Voor WorkFusion is het dan weer een ander verhaal. WorkFusion werkt goed voor zeer basis workflows die weinig tot geen logica nodig hebben. Van zodra dit criteria overschreven wordt, is het rap zeer moeilijk om een geschikte implementatie te voorzien. Het gebruik van basis Groovy zonder mogelijke uitbreidingen en het ontbreken van syntax highlighting en autocomplete beperken de mogelijkheid het schrijven van eigen activiteiten sterk.

[...]

%%=============================================================================
%% Bijlagen
%%=============================================================================

\appendix
\renewcommand{\chaptername}{Appendix}

%%---------- Onderzoeksvoorstel -----------------------------------------------

\chapter{Onderzoeksvoorstel}

Het onderwerp van deze bachelorproef is gebaseerd op een onderzoeksvoorstel dat vooraf werd beoordeeld door de promotor. Dat voorstel is opgenomen in deze bijlage.

% Verwijzing naar het bestand met de inhoud van het onderzoeksvoorstel
%---------- Inleiding ---------------------------------------------------------

\section{Introductie} % The \section*{} command stops section numbering
\label{sec:introductie}
GraphQL is al genoeg vergeleken geweest met REST API's. De volgende stap echter, blijkt  nog niet besproken. Hoe moet GraphQL gecombineerd worden met niet conventionele graaf databanken? 
Het onderzoek bestaat dan ook uit volgende vragen:
\begin{itemize}
	\item Hoe matuur is het gebruik van GraphQL (+-) in combinatie met graaf databanken? 
	\item Kunnen deze technologieën makkelijk gecombineerd worden?
	\item Zijn ze onderhoudbaar en schaalbaar?
	\item Hoe zit het met de performantie?
	\item Hoe stel ik een zoekboom bloot in GraphQL attributen?
\end{itemize}

Uit dit onderzoek moet blijken of het voor bedrijven de tijd waard is om GraphQL te combineren en te gebruiken in combinatie met niet conventionele graaf databanken.
% bevat:
 % de probleemstelling en context
 % de motivatie en relevantie voor het onderzoek
 % de doelstelling en onderzoeksvraag/-vragen

%---------- Stand van zaken ---------------------------------------------------

\section{Literatuurstudie}
\label{sec:literatuurstudie}
De wiskundige definitie van een graaf gaat als volgt, een graaf G bestaat uit een verzameling knopen V, en een verzameling bogen E. Elke boog verbindt twee knopen, en we noteren e = (v,w). De graaf G wordt genoteerd als het koppel (V,E), dus G = (V,E). \autocite{pod2Cursus}\\
Grafen zijn overal terug te vinden. Van het GPS (Global Positioning System) tot het netwerk van vrienden op sociale media. De eenvoudigste manier om een graaf structuur op te slaan is dan ook in een graaf databank. \\
Een graaf databank management systeem (graaf databank) is een online database management systeem met Create, Read, Update en Delete (CRUD) methoden die een graaf datamodel blootstellen. Graaf databanken zijn vooral gemaakt voor gebruik met online transactionele systemen (OLTP). Hierdoor zijn ze geoptimaliseerd voor transactionele performantie en gemaakt met transactionele integriteit en operationele beschikbaarheid in gedachten.

\subsection{GraphQL}
 GraphQL is een querytaal ontwikkeld door Facebook in 2012 en open source gemaakt in 2015. Sinds de release is de populariteit en het gebruik ervan steeds aan het stijgen. Zaken zoals 'get what you ask for' en het flexibel opbouwen van het databank schema zorgen ervoor dat GraphQL op korte tijd populair geworden is. Deze technologie kan bovenop een REST (Representational State Transfer) API (Application Programming Interface) service gebruikt worden maar velen zien het juist als een vervangmiddel voor REST API's. \autocite{dgraphDocs} Het feit dat GraphQL ondersteund wordt door de meest gebruikte programmeer talen zoals JavaScript, Python en Java \autocite{top10Lang} speelt in het voordeel van de technologie.
 
 \subsubsection{Get what you ask for}
 Het 'get what you ask for' principe werkt als volgt: in plaats van een volledig JSON object terug te krijgen, krijg je alleen die velden terug die gespecificeerd worden in de GraphQL query. Zo verminderd de grootte van data die doorgestuurd wordt over het netwerk wat resulteert in snellere response tijden. \autocite{gqlDocs}

\subsection{GraphQL +-}
GraphQL +- is een specificatie van GraphQL gericht op het aanspreken van graaf databanken. De queries worden opgebouwd aan de hand van zoek criteria en overeenstemmende patronen in een graaf om nodes terug te vinden. Het resultaat van zo'n query is zelf een graaf.  Een belangrijke opmerking omtrent GraphQL +- nog in de maak is. Er wordt op regelmatige basis een update doorgevoerd met bugfixes, nieuwe features, verbetering van de performantie, ... . \autocite{dgraphDocs}
%GraphQL +- is gebaseerd op de gewone GraphQL maar dan met enige wijzigingen en simplificaties om de taal te optimaliseren voor het aanspreken en ophalen van data uit graaf databanken.\autocite{dgraphDocs}

%\subsection{Graaf databanken}
%Graaf databanken bevatten sterk geconnecteerde data.\autocite{graphDatabases} Om dit op te vragen via een REST API zal ofwel meerdere requests naar de server en dus meerdere calls naar de databank nodig zijn, ofwel zal dit een zeer dure en tijdsintensieve operatie zijn. \autocite{everythingGQL}

\subsection{De kracht van graaf databanken}
Graaf databanken voorzien een flexibel data model en een geoptimaliseerde data query methode voor een set van use-cases waarbij de snelheid en performantie sterk verbeterd is en waar de latency bij het opvragen van de data sterk verminderd ten opzichte van een relationele en NoSQL databanken. Deze use-case zijn natuurlijk werken met sterk geconnecteerde data. % Bij niet graaf gerichte OLTP systemen zou dit een zeer intensieve, multi-join query zijn. 
Naast de performantie biedt ook de flexibiliteit een pluspunt. Grafen zijn natuurlijk additief. Dit wil zeggen dat gemakkelijk nieuwe relaties, nodes en labels toegevoegd kunnen worden.
Ook de beweeglijkheid van de data speelt een voordeel. We willen onze data laten evolueren met eenzelfde snelheid als het iteratief en incrementeel (agile) werkproces. De moderne graaf databanken zijn gebouwd voor het wrijvingsloos ondersteunen van development en onderhoud van het systeem. \autocite{graphDatabases}

\subsection{Combinatie}
GraphQL speelt als het ware een laag, een centraal aanspreek punt, tussen de client applicatie en de databank.  Hieruit kan de hele databank gemakkelijk aangesproken worden. Dit werkt in het voordeel van een ontwikkelaar die nu gemakkelijk verschillende delen van de databank tegelijk kan aanspreken.
%waarbij de verschillende tabellen tegelijk aangesproken kunnen worden en dus gemakkelijk gecombineerd worden tot 1 request waar alles inzit dat opgevraagd wordt, in tegenstelling tot REST API's. \autocite{everythingGQL}


% bevat:
% Hier beschrijf je de \emph{state-of-the-art} rondom je gekozen onderzoeksdomein. Dit kan bijvoorbeeld een literatuurstudie zijn. Je mag de titel van deze sectie ook aanpassen (literatuurstudie, stand van zaken, enz.). Zijn er al gelijkaardige onderzoeken gevoerd? Wat concluderen ze? Wat is het verschil met jouw onderzoek? Wat is de relevantie met jouw onderzoek?

%Verwijs bij elke introductie van een term of bewering over het domein naar de vakliteratuur, bijvoorbeeld~\autocite{Doll1954}! Denk zeker goed na welke werken je refereert en waarom.

% Voor literatuurverwijzingen zijn er twee belangrijke commando's:
% \autocite{KEY} => (Auteur, jaartal) Gebruik dit als de naam van de auteur
%   geen onderdeel is van de zin.
% \textcite{KEY} => Auteur (jaartal)  Gebruik dit als de auteursnaam wel een
%   functie heeft in de zin (bv. ``Uit onderzoek door Doll & Hill (1954) bleek
%   ...'')

%Je mag gerust gebruik maken van subsecties in dit onderdeel.

%---------- Methodologie ------------------------------------------------------
\section{Methodologie}
\label{sec:methodologie}

Om op de hierboven geschreven vragen een antwoord te bieden ga ik gebruik maken van simulaties en experimenten om productie waardige omgevingen na te bootsen. Hierbij wordt eerst gestart met een simpele DGraph graaf databank die aangesproken wordt met een Node.JS backend via GraphQL(+-). Stilaan wordt de databank uitgebreid tot een applicatiewaardige graaf databank. Hierbij  zal een onderzoek uitgevoerd worden naar onder andere de performantie, snelheid en onderhoudbaarheid van de databank en de GraphQL aanspreek methode. Nadien zal dit onderzoek uitgebreid worden naar andere graaf databanken zoals bijvoorbeeld ArrangoDB, Neo4J of FaunaDB.\\
Om de combinatie tussen GraphQL en graaf databanken te onderzoeken zal gebruik gemaakt worden van: 
\begin{enumerate}
	\item GraphQL (+-)
	\item DGraph Sandbox
\end{enumerate}
Hierbij wordt gekeken hoe lang het duurt, met opzoekingswerk inbegrepen om succesvol een lokale databank in docker te draaien via DGraph en via GraphQL +- er data uit op te halen.\\
Om schaalbaarheid en performantie te testen zal een testomgeving opgezet worden met volgende eigenschappen:
\begin{itemize}
	\item Een Node.JS backend gecombineerd met Apollo en GraphQL
	\item Een DGraph datbank met substantiële hoeveelheden data
\end{itemize}
Binnen dit experiment zal geleidelijk aan het data model en de data binnen de databank uitgebreid worden.

%---------- Verwachte resultaten ----------------------------------------------
\section{Verwachte resultaten}
\label{sec:verwachte_resultaten}
Aangezien de focus van GraphQL ligt op het makkelijk opstellen van queries voor developers, wordt er verwacht dat het gebruik er van in combinatie met graaf databanken geen probleem geeft. \\
De research dat gaat in het opzetten van beide technologieën blijft beperkt en is gemakkelijk op te zetten. Ook de schaalen beide technologieën goed mee met elkaar door onder andere het flexibele data model van GraphQL en de additieve eigenschap van graaf databanken. Het blootstellen van zoekbomen aan de hand van GraphQL attributen wordt gemakkelijk gemaakt door de GraphQL +- specificatie.

% Hier beschrijf je welke resultaten je verwacht. Als je metingen en simulaties uitvoert, kan je hier al mock-ups maken van de grafieken samen met de verwachte conclusies. Benoem zeker al je assen en de stukken van de grafiek die je gaat gebruiken. Dit zorgt ervoor dat je concreet weet hoe je je data gaat moeten structureren.

%---------- Verwachte conclusies ----------------------------------------------
\section{Verwachte conclusies}
\label{sec:verwachte_conclusies}
Er wordt verwacht dat de opzet van een project met GraphQL en graaf databanken bedrijven niet zo mogen tegenhouden om er mee te beginnen. Ook dat de leercurve voor GraphQL (+-) in combinatie met graaf databanken zou geen struikelblok mogen zijn en dat het de tijd nodig om dit allemaal te leren het waard is.

% Hier beschrijf je wat je verwacht uit je onderzoek, met de motivatie waarom. Het is \textbf{niet} erg indien uit je onderzoek andere resultaten en conclusies vloeien dan dat je hier beschrijft: het is dan juist interessant om te onderzoeken waarom jouw hypothesen niet overeenkomen met de resultaten.



%%---------- Andere bijlagen --------------------------------------------------
%%=============================================================================
%% Prijs vergelijking
%%=============================================================================
\chapter{\IfLanguageName{dutch}{Prijs vergelijking}{Price comparison}}

Met dank aan Robonext\footnote{https://robonext.eu/} voor het voorzien van deze afbeeldingen.

Niet elke provider heeft tijdig gereageerd op de mail met vragen rond de prijs van hun \acrshort{rpa} oplossing.

\section{Marktonderzoek}
De \acrshort{rpa} markt is verdeeld in vier delen. De 'Big Three', UiPath, Automation Anywhere en BluePrism, nemen ongeveer 70\% van de hele markt op, de overige 30\% wordt opgesplitst onder de rest van de kleinere providers. Hierbij kan een ruwe benadering gevonden worden waarbij UiPath ongeveer 35\% van de markt in zijn bezit heeft, Automation Anywhere dan weer 25\% en tot slot Blueprism die ongeveer 10\% in zijn bezit heeft. \autocite{marktRPA}

Ook kan hierbij de assumptie gemaakt worden dat de hoop van kleine spelers minder en minder relevant worden en dat het uiteindelijk op een competitieve race tussen de 'Big 3' zal uitdraaien. \autocite{marktRPA}

\subsection{'The Big Three'}
Uit de verschillende foto's kan nog maar eens besloten worden dat UiPath het meeste marktaandeel heeft van alle providers en dat de markt gedomineerd wordt door de grote drie spelers binnen de markt. 
\begin{figure}[h!]
	\includegraphics[width=\linewidth]{pricing/price_comparison_1.png}
	\caption[Cloud editie benchmarks]{Het verschil in prijs voor een cloud oplossing bij de grote 3 RPA providers.}
	\label{fig:price_1}
\end{figure}
\begin{figure}[h!]
	\includegraphics[width=\linewidth]{pricing/price_comparison_2.png}
	\caption[Grote 3 Revenue Streams]{Schatting naar marktwaarden en marktaandeel van de grote 3 ten opzichte van de rest.}
	\label{fig:price_2}
\end{figure}
\begin{figure}[h!]
	\includegraphics[width=\linewidth]{pricing/price_comparison_3.png}
	\caption[Benchmarkings]{De marktleiders en de opkomende spelers.}
	\label{fig:price_3}
\end{figure}

\subsection{Microsoft Flow}
Als niet traditionele \acrshort{rpa} provider wil Microsoft zich ook competitief op de markt zetten met Flow die een sterken samenhang kent met Power Apps en Azure. Hierbij wordt gewerkt met licenties en subscripties.
\begin{figure}[h!]
	\includegraphics[width=\linewidth]{pricing/price_mf_1.png}
	\caption[Microsoft Flow licenties]{Prijs van de verschillende licenties van Microsoft Flow.}
	\label{fig:price_mf_1}
\end{figure}

\begin{figure}[h!]
	\includegraphics[width=\linewidth]{pricing/price_mf_2.png}
	\caption[Microsoft Flow gebruikersplan]{Prijs voor de verschillende gebruikersplannen bij Microsoft Flow.}
	\label{fig:price_mf_2}
\end{figure}

\begin{figure}[h!]
	\includegraphics[width=\linewidth]{pricing/price_mf_3.png}
	\caption[Microsoft Flow extra services]{Prijs voor extra services bij Microsoft Flow.}
	\label{fig:price_mf_3}
\end{figure}

%%---------- Referentielijst --------------------------------------------------

\printbibliography[heading=bibintoc]

\end{document}
